\section{Metodología de desarrollo}
La fase de desarrollo del proyecto ``Autogestión SENA'' se caracterizó por la integración constante de herramientas de Inteligencia Artificial (IA) para mejorar la eficiencia y la colaboración del equipo. A continuación, se detallan los hitos clave y la participación de la IA:

\subsection{Fase 1: Diseño y Arquitectura}
En esta etapa inicial, la IA actuó como un consultor técnico y creativo para establecer las bases del proyecto:

\begin{itemize}
    \item \textbf{Diseño de Datos:} Se llevó a cabo la elaboración del Modelo Entidad-Relación utilizando la herramienta Draw.io. Este modelo definió la estructura de la base de datos necesaria para la autogestión de usuarios y asignaciones.
    
    \item \textbf{Diseño de Interfaz (Mockup):} Se creó el diseño del mockup en Figma. En este proceso, se utilizó la IA para buscar inspiración de diseños modernos e intuitivos, asegurando una experiencia de usuario (UX) óptima.
    
    \item \textbf{Definición de Arquitectura:} Se estableció la arquitectura general del proyecto (Web, Mobile y Back-end). La IA fue fundamental para consultar y validar diferentes modelos de arquitectura, asegurando que la elección fuera robusta y escalable para las necesidades del SENA.
\end{itemize}

La \cref{fig:inicio_proyecto} presenta el flujo completo del proyecto desde su inicio hasta las fases de aprendizaje y mantenibilidad.

\begin{figure}[htbp]
    \centering
    \includegraphics[width=0.48\textwidth]{graphics/inicio_proyecto.png}
    \caption{Inicio del proyecto y fases de desarrollo con integración de IA}
    \label{fig:inicio_proyecto}
\end{figure}

\subsection{Fase 2: Desarrollo e Implementación Asistida}
La IA se integró directamente en las tareas de codificación y debugging para potenciar la productividad individual y la calidad del código:

\begin{itemize}
    \item \textbf{Desarrollo del Back-end:} Se trabajó en la creación de los endpoints necesarios. La IA ayudó en el desarrollo de endpoints en Django/Python, actuando como un asistente para generar el esqueleto de las funciones y la lógica inicial.
    
    \item \textbf{Desarrollo del Front-end:} Se desarrolló la interfaz de usuario. Se utilizó la ayuda de la IA para el desarrollo de animaciones en la parte web, logrando un diseño moderno sin invertir tiempo excesivo en código complejo.
    
    \item \textbf{Conexión de Componentes:} Se realizó la conexión del Front-end con el Back-end, asegurando la comunicación correcta para la transferencia de datos.
\end{itemize}

La \cref{fig:ciclo_desarrollo} ilustra el ciclo de desarrollo asistido por IA, mostrando la interacción continua entre las diferentes etapas del proyecto.

\begin{figure}[htbp]
    \centering
    \includegraphics[width=0.45\textwidth]{graphics/ciclo_desarrollo.png}
    \caption{Ciclo de desarrollo asistido por IA}
    \label{fig:ciclo_desarrollo}
\end{figure}

\subsection{Fase 3: Aprendizaje y Mantenibilidad del Código}
Un aspecto crucial de la colaboración con la IA fue su impacto en el aprendizaje y la calidad del producto final:

\begin{itemize}
    \item \textbf{Soporte de Sintaxis y Mantenimiento:} La IA proporcionó ayuda constante con la sintaxis del código y su mantenibilidad. Esto redujo errores y aseguró que el código fuera limpio y fácil de entender por todos los colaboradores.
    
    \item \textbf{Resolución de Problemas:} Se utilizó la IA para consultas de errores y el porqué de los fallos (debugging). Esto ofreció soluciones rápidas, minimizando el tiempo de bloqueo del equipo.
    
    \item \textbf{Explicación de Funciones:} Para acelerar el proceso de conocimiento interno, la IA proporcionó explicaciones de funciones tanto en el Front-end como en el Back-end, facilitando la comprensión de partes complejas del código entre los miembros del equipo.
\end{itemize}

Este enfoque demostró cómo la IA no solo aceleró la entrega del proyecto sino que también sirvió como una plataforma de aprendizaje continuo para el equipo, elevando las habilidades técnicas de todos los colaboradores.