\section{Metodología}

La fase de desarrollo del proyecto "Autogestión SENA" integró técnicas ágiles con herramientas de inteligencia artificial (IA) como apoyo para acelerar tareas y mejorar la calidad. A continuación se describen las metodologías y prácticas empleadas durante el proyecto.

\subsection{Metodología Ágil (Scrum)}

El equipo trabajó bajo el marco de trabajo Scrum durante cuatro meses. Se planificó inicialmente una cadencia de 2 sprints por mes, resultando en un total de 7 sprints ejecutados en el ciclo de desarrollo (debido a ajustes y entregas intermedias). Las ceremonias y la forma de trabajo fueron:
\begin{itemize}
	\item Sprint Planning: definición de objetivos y selección de historias de usuario para cada sprint.
	\item Daily Stand-ups: reuniones diarias de 30 minutos para seguimiento del progreso y bloqueo.
	\item Sprint Review: demostraciones del trabajo entregado y validación con instructores y stakeholders.
	\item Sprint Retrospective: identificación de mejoras y acciones a implementar en el siguiente sprint.
\end{itemize}

La práctica del Scrum permitió priorizar requisitos, mantener iteraciones cortas y recibir feedback temprano de los instructores, lo cual fue crucial para validar decisiones técnicas y de diseño.

\subsubsection{Cronograma de sprints}
Se realizó una planificación de 7 sprints durante 4 meses. En la Tabla~\ref{tab:sprints} se listan las fechas aproximadas y los entregables principales de cada sprint.

\begin{table}[htbp]
	\centering
	\caption{Resumen de Sprints (ejemplo)}
	\label{tab:sprints}
	\begin{tabular}{lll}
		oprule
		extbf{Sprint} & \textbf{Fechas} & \textbf{Objetivo/Entregable} \\
		\midrule
		Sprint 1 & 2025-05-01 -- 2025-05-14 & Modelado dominio, base de datos, prototipo inicial \\
		Sprint 2 & 2025-05-15 -- 2025-05-28 & Endpoints básicos y estructura del backend \\
		Sprint 3 & 2025-06-01 -- 2025-06-14 & Front-end inicial y conexión a API \\
		Sprint 4 & 2025-06-15 -- 2025-06-28 & Funcionalidad de asignación y pruebas unitarias \\
		Sprint 5 & 2025-07-01 -- 2025-07-14 & Integración móvil y UX mejorada \\
		Sprint 6 & 2025-07-15 -- 2025-07-28 & Pruebas de integración y calidad, refactorizaciones \\
		Sprint 7 & 2025-08-01 -- 2025-08-14 & Despliegue en staging, documentación y entrega final \\
		\bottomrule
	\end{tabular}
\end{table}

\subsection{Diseño y planificación}

El proceso de diseño inició con la definición del modelo de datos y mockups de la interfaz. Se emplearon herramientas como Draw.io para diagramas (ERD y arquitectónicos) y Figma para prototipado visual y pruebas de usabilidad.

La base de datos elegida fue MySQL, y la estructura del modelo de dominio incluyó las entidades principales: \emph{Usuario, Persona, Rol, Aprendiz, Instructor, Horario, Solicitud, Sesión}. La IA se utilizó para sugerir normalizaciones, índices y tests conceptuales para mejorar la integridad y el rendimiento.

\begin{figure}[htbp]
	\centering
	\includegraphics[width=0.7\linewidth]{graphics/comparacion_metodologias.pdf}
	\caption{Comparación de metodologías ágiles (Scrum, Kanban, XP)}
	\label{fig:comparacion_metodologias}
\end{figure}

\subsection{Implementación asistida por IA}

Durante la fase de implementación, la IA fue usada para generar esqueletos de endpoints en Django, proponer plantillas de validación, sugerir pruebas unitarias e indicar posibles refactorizaciones basadas en patrones de diseño. No obstante, cada sugerencia fue revisada por el equipo antes de su integración.

\subsection{Recolección de información para despliegue}
La recolección de información sobre el despliegue y configuración se realizó principalmente mediante preguntas estructuradas a los instructores: requisitos de infraestructura, dependencias, credenciales y políticas de seguridad. Esta información se consolidó en una checklist y se utilizó para generar scripts y comandos de despliegue con la ayuda de la IA, que luego se ajustaron y validaron por el equipo.

\subsection{Explicación de funciones y transferencia de conocimiento}

La IA ayudó a documentar y explicar funciones complejas en lenguaje natural, generando diagramas de flujo y pseudocódigo que facilitaron la transferencia de conocimiento entre miembros del equipo. Esto permitió una incorporación rápida de desarrolladores con menos experiencia.

\subsection{Soporte con el código y resolución de problemas}
La IA sirvió como soporte para resolver dudas de sintaxis, proponer correcciones y acelerar la solución de errores. Esto resultó en código más limpio y en una reducción relativa del tiempo de bloqueo del equipo. Para cada cambio crítico sugerido por la IA, se mantuvo un control de revisión humana y pruebas automatizadas.

\subsection{Patrones de diseño y principios SOLID}

El uso de la IA permitió identificar fragmentos de código con múltiples responsabilidades y sugerir la aplicación de principios SOLID. Entre las acciones concretas realizadas se incluyen:
\begin{enumerate}
	\item Separación de responsabilidades mediante Service y Repository layers.
	\item Definición de interfaces para apoyar el Dependency Inversion y facilitar pruebas.
	\item Refactorización para simplificar condicionales y mejorar coherencia del código.
\end{enumerate}

Estas decisiones favorecieron la mantenibilidad y la testabilidad del sistema.