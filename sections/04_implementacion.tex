\section{Implementación}
\label{sec:implementacion}

La fase de desarrollo del proyecto "Autogestión SENA" se caracterizó por la implementación constante de herramientas de Inteligencia Artificial (IA) para mejorar la eficiencia y la colaboración del equipo. A continuación, se detallan los pasos clave y la participación de la IA durante los procesos del desarrollo del proyecto:

1. Fase de Diseño y Planificación en Equipo
En esta primera etapa, usamos la inteligencia artificial como un apoyo para planear el proyecto :
Diseño de la Base de Datos: Entre todos creamos el modelo de base de datos usando Draw.io. Nos ayudamos para definir cómo organizaríamos la información de los usuarios y sus actividades.
Diseño de la Interfaz (Mockup): Diseñamos el prototipo de la aplicación en Figma, usamos la IA para buscar inspiración y elegir entre todos un diseño moderno y fácil de entender para los usuarios.

Planificación de la Arquitectura: se estructuro el proyecto (la parte Web, Móvil y el backend) con buenas practicas de desarrollo y pues se uso la IA para comparar opciones de estructuracion y parte del desarrollo del codigo y asi asegurarnos de crear una base sólida para el sistema del SENA.

2. Desarollo y Programación con Apoyo Mutuo

Utilizamos la IA como una herramienta más del equipo para programar y resolver problemas:

Desarrollo del Back-end:
La IA nos apoyó escribiendo la base del código en Django/Python, lo que nos permitió avanzar más rápido como equipo.

Desarrollo de la Interfaz (Front-end): Al trabajar en la apariencia de la aplicación, usamos la IA entre varios para crear animaciones atractivas sin complicarnos demasiado, logrando un resultado profesional.

Conexión entre Front-end y Back-end: Integramos la parte visual con el backend colaborando para que la comunicación y el paso de datos funcionaran correctamente.

3. Aprendizaje en Grupo y Código Mantenible

La IA nos sirvió para aprender y mejorar la calidad de nuestro trabajo:

Soporte con el Código: La IA nos fue de gran ayuda para resolver dudas de sintaxis. Esto nos permitió escribir un código más limpio y comprensible para todo el equipo, facilitando el mantenimiento y flujo del sitema.

Resolución de Problemas: Se utilizó la IA para consultas de errores y el porqué de los fallos (debugging). Esto ofreció soluciones rápidas, reduciendo el tiempo de bloqueo del equipo.

Explicación de Funciones: Para acelerar el proceso de conocimiento interno, la IA proporcionó explicaciones de funciones tanto en el Front-end como en el Back-end, facilitando la comprensión de partes complejas del código entre los miembros del equipo.

Este enfoque demostró cómo la IA no solo aceleró la entrega del proyecto sino que también sirvió como una plataforma de aprendizaje continuo para el equipo, elevando las habilidades técnicas de todos los participantes de este proyecto.

\subsection{Implementación y generación de código asistida por IA}
\begin{itemize}
	\item \textbf{Generación de esqueletos y endpoints}: la IA se empleó para crear plantillas de endpoints en Django (por ejemplo, POST /api/instructores, GET /api/rol), validaciones y serializadores básicos. Estas plantillas se adaptaron y ampliaron por el equipo.
	\item \textbf{Refactorización y patrones}: la IA ayudó a identificar oportunidades de refactorización, proponiendo la extracción de servicios y la organización por capas (Repository, Service Layer) para separar la lógica de negocio de la persistencia.
	\item \textbf{Explicación de funciones y mapeo de flujos}: la IA generó descripciones y pseudocódigo para funciones complejas y ayudó a mapear interacciones entre UI y API, acelerando la transferencia de conocimiento entre miembros del equipo.
	\item \textbf{Front-end y accesibilidad}: en la parte visual, la IA sugirió componentes, animaciones y estilos con Tailwind; el equipo supervisó las recomendaciones para asegurar accesibilidad y coherencia visual.
	\item \textbf{Contratos API y pruebas}: se definieron requests y responses con ejemplos y casos de prueba generados por IA para validar integraciones y evitar discrepancias.
	\item \textbf{Soporte para el aprendizaje y mantenimiento}: los participantes del proyecto adquirimos explicaciones y ejemplos explicativos que facilitaron su incorporación y la mantenibilidad del código a largo plazo.
\end{itemize}

\subsection{Implementación Técnica Detallada}

\subsubsection{Backend Django REST Framework}
El backend se implementó utilizando Django 4.2 con Django REST Framework para la construcción de APIs RESTful. Las características principales incluyen:

\begin{itemize}
\item \textbf{Modelos y Serializadores}: Se definieron modelos para las entidades principales (Usuario, Instructor, Aprendiz, AsignacionInstructor, VisitaSeguimiento) con validaciones personalizadas y serializadores para la transformación de datos JSON.
\item \textbf{Autenticación y Autorización}: Implementación de JWT (JSON Web Tokens) para autenticación stateless, con permisos basados en roles (Administrador-Coordinador, Instructor, Aprendiz, Operador Sofia Plus).
\item \textbf{Endpoints Principales}:
  \begin{itemize}
  \item \texttt{POST /api/auth/login}: Autenticación de usuarios
  \item \texttt{GET /api/instructores}: Listado de instructores disponibles
  \item \texttt{POST /api/asignaciones}: Creación de asignaciones instructor-aprendiz
  \item \texttt{GET /api/visitas}: Seguimiento de visitas de seguimiento
  \end{itemize}
\item \textbf{Base de Datos MySQL}: Configuración con índices optimizados para consultas de asignación y reporting.
\end{itemize}

\subsubsection{Frontend React + TypeScript}
La interfaz web se desarrolló con React 18 y TypeScript, utilizando Vite como bundler para desarrollo rápido:

\begin{itemize}
\item \textbf{Gestión de Estado}: Implementación de Context API para manejo de estado global de autenticación y datos de usuario.
\item \textbf{Componentes Principales}:
  \begin{itemize}
  \item Dashboard administrativo con métricas en tiempo real
  \item Formularios de asignación con validación en tiempo real
  \item Tablas de datos con filtrado y paginación
  \item Perfiles de usuario con gestión de roles
  \end{itemize}
\item \textbf{UI/UX}: Diseño responsivo con Tailwind CSS, siguiendo principios de accesibilidad WCAG 2.1.
\item \textbf{Integración API}: Cliente HTTP personalizado con interceptores para manejo de tokens JWT y errores.
\end{itemize}

\subsubsection{Aplicación Móvil .NET MAUI}
La aplicación móvil multiplataforma se desarrolló con .NET MAUI para Android e iOS:

\begin{itemize}
\item \textbf{Arquitectura MVVM}: Patrón Model-View-ViewModel para separación de responsabilidades.
\item \textbf{Funcionalidades Móviles}:
  \begin{itemize}
  \item Consulta de asignaciones activas
  \item Registro de visitas de seguimiento con geolocalización
  \item Notificaciones push para nuevas asignaciones
  \item Modo offline con sincronización automática
  \end{itemize}
\item \textbf{Integración Nativa}: Acceso a GPS, cámara y notificaciones del dispositivo.
\end{itemize}

\subsubsection{Despliegue con Docker}
La contenerización se implementó con Docker Compose para entornos de desarrollo y producción:

\begin{itemize}
\item \textbf{Servicios Contenedorizados}:
  \begin{itemize}
  \item \texttt{web}: Aplicación Django + Gunicorn
  \item \texttt{db}: MySQL 8.0 con volúmenes persistentes
  \item \texttt{frontend}: Servidor Nginx para archivos estáticos de React
  \end{itemize}
\item \textbf{Orquestación}: Docker Compose para desarrollo local con hot-reload.
\item \textbf{Configuración}: Variables de entorno para diferentes entornos (desarrollo, staging, producción).
\end{itemize}

\subsection{Generación de recursos y reproducibilidad}
Todas las gráficas y tablas incluidas en este artículo se generan mediante el script \texttt{code/generate\_figures.py}. Para reproducir las gráficas:
\begin{enumerate}
	\item Instalar dependencias del proyecto: \texttt{pip install -r requirements.txt}.
	\item Ejecutar: \texttt{python code/generate\_figures.py}.
\end{enumerate}
Los archivos se almacenan en la carpeta \texttt{graphics/} y la tabla en \texttt{tables/frameworks\_comparison.tex}.



