\section{Desarrollo}

La fase de desarrollo del proyecto "Autogestión SENA" se caracterizó por la inplementacion constante de herramientas de Inteligencia Artificial (IA) para mejorar la eficiencia y la colaboración del equipo. A continuación, se detallan los pasos clave y la participación de la IA durante los proceso del desarrollo del proyecto:

1. Fase de Diseño y Planificación en Equipo
En esta primera etapa, usamos la inteligencia artificial como un apoyo para planear el proyecto :
Diseño de la Base de Datos: Entre todos creamos el modelo de base de datos usando Draw.io. Nos ayudamos para definir cómo organizaríamos la información de los usuarios y sus actividades.
Diseño de la Interfaz (Mockup): Diseñamos el prototipo de la aplicación en Figma, usamos la IA para buscar inspiración y elegir entre todos un diseño moderno y fácil de entender para los usuarios.

Planificación de la Arquitectura: se estructuro el proyecto (la parte Web, Móvil y el backend) con buenas practicas de desarrollo y pues se uso la IA para comparar opciones de estructuracion y parte del desarrollo del codigo y asi asegurarnos de crear una base sólida para el sistema del SENA.

2. Desarollo y Programación con Apoyo Mutuo

Utilizamos la IA como una herramienta más del equipo para programar y resolver problemas:

Desarrollo del Back-end:
La IA nos apoyó escribiendo la base del código en Django/Python, lo que nos permitió avanzar más rápido como equipo.

Desarrollo de la Interfaz (Front-end): Al trabajar en la apariencia de la aplicación, usamos la IA entre varios para crear animaciones atractivas sin complicarnos demasiado, logrando un resultado profesional.

Conexión entre Front-end y Back-end: Integramos la parte visual con el backend colaborando para que la comunicación y el paso de datos funcionaran correctamente.

3. Aprendizaje en Grupo y Código Mantenible

La IA nos sirvió para aprender y mejorar la calidad de nuestro trabajo:

Soporte con el Código: La IA nos fue de gran ayuda para resolver dudas de sintaxis. Esto nos permitió escribir un código más limpio y comprensible para todo el equipo, facilitando el mantenimiento y flujo del sitema.

Resolución de Problemas: Se utilizó la IA para consultas de errores y el porqué de los fallos (debugging). Esto ofreció soluciones rápidas, reduciendo el tiempo de bloqueo del equipo.

Explicación de Funciones: Para acelerar el proceso de conocimiento interno, la IA proporcionó explicaciones de funciones tanto en el Front-end como en el Back-end, facilitando la comprensión de partes complejas del código entre los miembros del equipo.

Este enfoque demostró cómo la IA no solo aceleró la entrega del proyecto sino que también sirvió como una plataforma de aprendizaje continuo para el equipo, elevando las habilidades técnicas de todos los participantes de este proyecto.

