\section{Implementación del software}
La implementación de ``Autogestión SENA'' requirió decisiones tecnológicas estratégicas y una metodología de desarrollo colaborativa asistida por IA.

\subsection{Stack tecnológico}
Se seleccionó un stack full-stack moderno:
\begin{itemize}
    \item \textbf{Back-end:} Python + Django para APIs REST y lógica de negocio
    \item \textbf{Front-end Web:} React + Vite + Tailwind CSS para interfaces responsivas y modernas
    \item \textbf{Front-end Mobile:} .NET MAUI para aplicación móvil multiplataforma
    \item \textbf{Base de datos:} Modelo relacional optimizado para gestión de instructores y asignaciones
    \item \textbf{Herramientas de diseño:} Draw.io (modelado ER), Figma (mockups de interfaz)
\end{itemize}

\subsection{Proceso de implementación asistido por IA}
La IA se integró en cada fase del desarrollo:

\textbf{Generación de código base:} La IA generó estructuras iniciales de endpoints en Django, reduciendo el tiempo de desarrollo de código repetitivo.

\textbf{Desarrollo de componentes visuales:} Se utilizó asistencia de IA para implementar animaciones y efectos visuales en el front-end, logrando interfaces modernas sin conocimiento profundo previo de CSS avanzado.

\textbf{Debugging interactivo:} Cuando surgían errores, la IA proporcionaba explicaciones detalladas del problema y sugerencias de solución, acelerando significativamente la resolución de bugs.

\textbf{Documentación y explicación de código:} La IA facilitó la comprensión de código complejo al proporcionar explicaciones claras de funciones tanto del back-end como del front-end, mejorando el aprendizaje colaborativo del equipo.
