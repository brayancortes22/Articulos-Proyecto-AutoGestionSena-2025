\section{Implementación}
Para construir la solución completa (parte interna y externa) se emplearon herramientas y buenas prácticas que facilitaron la implementación y la entrega continua.

\subsection{Tecnologías principales}
- \textbf{Back-end}: Python con Django y Django REST Framework para endpoints REST.
- \textbf{Base de datos}: MySQL como motor de persistencia.
- \textbf{Front-end web}: React, Vite y Tailwind CSS para componentes y estilos.
- \textbf{Aplicación móvil}: .NET MAUI para aplicaciones iOS/Android; durante el desarrollo se priorizó la plataforma Android para pruebas y despliegue inicial.
- \textbf{CI/CD y contenedores}: Docker y docker-compose para entornos reproducibles y plantillas de despliegue.

\subsection{Contratos de API y pruebas}
Se definieron contratos claros de API (requests/responses) y se implementaron pruebas de integración automáticas que verificaron campos obligatorios, validaciones y códigos de respuesta. La IA ayudó a generar ejemplos de payloads, tests unitarios y casos de prueba.

\begin{table}[htbp]
\centering
\caption{Comparación de Frameworks de Desarrollo Web}
\label{tab:frameworks}
\begin{tabular}{lccccc}
\toprule
\textbf{Framework} & \textbf{Lenguaje} & \textbf{Performance} & \textbf{Curva Aprendizaje} & \textbf{Comunidad} & \textbf{Puntuación} \\
\midrule
React & JavaScript & Alta & Media & Excelente & 9.2 \\
Angular & TypeScript & Alta & Alta & Excelente & 8.7 \\
Vue.js & JavaScript & Alta & Baja & Buena & 8.9 \\
Django & Python & Media & Media & Excelente & 8.5 \\
Spring Boot & Java & Alta & Alta & Excelente & 8.8 \\
Laravel & PHP & Media & Baja & Buena & 8.1 \\
Express.js & JavaScript & Alta & Baja & Buena & 8.3 \\
\bottomrule
\end{tabular}
\end{table}

\subsection{Flujo de integración y despliegue}
Se estableció un pipeline de CI que ejecuta pruebas unitarias y despliega en un entorno de staging con Docker Compose. Para el despliegue en producción se definieron scripts y una checklist que incluye variables de entorno, migraciones de base de datos y validación de servicios externos.

\subsection{Aportes de la IA en la implementación}
La IA fue usada como asistente para:
\begin{itemize}
	\item Generar esqueletos de endpoints con validaciones y serializadores {	tfamily (Django REST Framework)}.
	\item Generar esqueletos de endpoints con validaciones y serializadores {\ttfamily Django REST Framework}.
	\item Plantillas de pruebas unitarias y de integración para acelerar la cobertura de tests.
	\item Generar comandos y scripts de despliegue (ej.: Dockerfile y docker-compose) que luego fueron revisados por humanos.
\end{itemize}

\begin{figure}[htbp]
	\centering
	\includegraphics[width=0.9\linewidth]{graphics/sprint_timeline.pdf}
	\caption{Cronograma de sprints ejecutados durante el desarrollo (7 sprints en 4 meses).}
	\label{fig:sprint_timeline}
\end{figure}

\begin{figure}[htbp]
	\centering
	\includegraphics[width=0.9\linewidth]{graphics/uso_ia_barras.pdf}
	\caption{Distribución del uso de IA por tipo de tarea durante el proyecto.}
	\label{fig:uso_ia}
\end{figure}
