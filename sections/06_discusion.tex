\section{Discusión}
La experiencia de desarrollo de ``Autogestión SENA'' con asistencia de IA revela importantes implicaciones para la educación en programación y el desarrollo de software.

\subsection{Impacto en el aprendizaje colaborativo}
La IA actuó como un agente de aprendizaje colaborativo, permitiendo al equipo resolver dudas técnicas de manera inmediata sin depender de la disponibilidad de expertos humanos. Esto es particularmente relevante en contextos educativos como el SENA, donde los estudiantes necesitan apoyo continuo durante el proceso de aprendizaje \cite{cadena2023ensenanza}.

\subsection{Equilibrio entre automatización y pensamiento crítico}
Es fundamental destacar que la IA se utilizó para tareas mecánicas y repetitivas, mientras que el diseño, la lógica de negocio y las decisiones arquitectónicas permanecieron bajo control humano. Esta división de trabajo permitió que el equipo se concentrara en aspectos que requieren creatividad y pensamiento crítico \cite{teran2023implementacion}.

\subsection{Productividad y calidad del código}
La asistencia de IA aceleró significativamente el proceso de desarrollo, reduciendo el tiempo invertido en tareas repetitivas como la generación de estructuras básicas de código, endpoints y animaciones. Sin embargo, el código generado requirió supervisión y validación humana para garantizar su calidad y mantenibilidad \cite{pincay2025aplicaciones}.

\subsection{Implicaciones para la industria}
Los resultados sugieren que la integración estratégica de herramientas de IA puede transformar la forma en que los equipos de desarrollo abordan proyectos full-stack, especialmente en entornos educativos y de formación profesional. La clave está en utilizar la IA como complemento del talento humano, no como reemplazo.