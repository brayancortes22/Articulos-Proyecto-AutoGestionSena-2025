\section{Discusión}

La integración de IA en el desarrollo de "Autogestión SENA" plantea varios puntos de discusión relevantes para la enseñanza y la práctica profesional.

\subsection{Impacto en el aprendizaje colaborativo}
La IA facilitó el aprendizaje colectivo al permitir la resolución inmediata de dudas y la explicación de conceptos técnicos, beneficiando especialmente entornos educativos donde el acceso a tutores puede ser limitado.

\subsection{Equilibrio entre automatización y pensamiento crítico}
Es crucial mantener la IA en un rol complementario: delegar tareas mecánicas a la IA mientras las decisiones de arquitectura, diseño y lógica de negocio quedan bajo responsabilidad humana.

\subsection{Productividad y calidad del código}
El apoyo de la IA aceleró tareas repetitivas, pero el código generado requiere siempre revisión y adaptación por parte del equipo para garantizar calidad, seguridad y mantenibilidad.

\subsection{Implicaciones prácticas}
La experiencia sugiere que la integración estratégica de IA puede transformar flujos de trabajo en proyectos full-stack, especialmente en formación profesional, si se acompaña de buenas prácticas y supervisión humana.

\subsection{Complicaciones registradas}
Durante el ciclo se documentaron complicaciones como discrepancias en integraciones, diferencias entre entornos, y limitaciones en sugerencias de la IA. Para mitigar estos problemas, implementamos pruebas de integración, contenedores Docker, revisión humana y almacenamiento seguro de secretos.

\begin{figure}[htbp]
	\centering
	\includegraphics[width=0.85\linewidth]{graphics/correlacion_devops.pdf}
	\caption{Correlación entre automatización y performance del equipo (ilustración del impacto de prácticas DevOps).}
	\label{fig:correlacion_devops}
\end{figure}