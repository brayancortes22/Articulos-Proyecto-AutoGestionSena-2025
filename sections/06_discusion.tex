\section{Reflexión}

Al crear la app “Autogestión SENA” vimos claramente lo útil que es la Inteligencia Artificial. La IA nos ayudó mucho como herramienta y como compañero de aprendizaje. Tuvo dos grandes beneficios para el equipo:

Resolver dudas rápido
Hacer el trabajo más rápido
Resolver dudas y bloqueos al instante

El mayor problema en los proyectos en equipo es quedarse atascado esperando que alguien te ayude con una duda técnica. Con la IA eso desapareció: teníamos un “experto” disponible 24 horas.

Cuando teníamos un bug, una función difícil de escribir o queríamos un código de calidad, se lo preguntábamos a la IA.
Ella nos daba la solución al instante. y nos explicaba como funcionaba y el porque era mejor hacerlo de esa manera.

La IA creó rápido cosas como:
Estructuras básicas de las páginas web (endpoints en Django)
Animaciones bonitas en la parte visible (front-end)
Código repetitivo que normalmente toma mucho tiempo

Gracias a eso escribimos el código mucho más rápido. Tuvimos más tiempo para lo realmente importante: hablar de cómo hacer la app fácil de usar, cómo organizar bien la información y cómo mejorar la experiencia del usuario. Esas cosas sí las tiene que hacer una persona, y con la IA pudimos dedicarles más tiempo y energía.
En resumen: la IA no reemplazó al equipo, sino que nos quitó el trabajo aburrido y nos dejó enfocarnos en lo que de verdad importa.

