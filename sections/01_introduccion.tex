El SENA es una institución colombiana que forma a miles de personas en oficios técnicos. Tiene muchos centros de formación en diferentes lugares del país. Uno de los trabajos más importantes que hacen es asignar instructores a los estudiantes, pero antes lo hacían con correos, hojas de Excel y llamadas por teléfono, lo que causaba muchos errores y demoras.

Por eso reestructuramos el proyecto "Autogestión SENA", una aplicación que automatiza y mejora este proceso. La app web permite a los coordinadores asignar instructores de forma fácil, a los instructores ver sus horarios y tareas, y a los aprendices saber quién es su instructor. Tiene versión web y móvil, con diferentes permisos para cada tipo de usuario.

Reestructuramos el proyecto en 4 meses usando metodología ágil (Scrum), con 7 ciclos de trabajo cortos que nos permitieron probar y mejorar según la opinión de los usuarios reales. Usamos mucho la inteligencia artificial (sobre todo GitHub Copilot) para acelerar el desarrollo, creando código, arreglando problemas y documentando todo.

Este artículo cuenta toda la experiencia, desde la idea inicial hasta tener el sistema funcionando. Explica cómo la colaboración entre humanos y IA cambió la forma de hacer software. Incluye detalles técnicos, decisiones que tomamos, funciones principales y resultados obtenidos.