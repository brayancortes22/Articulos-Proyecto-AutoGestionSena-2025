\section{Introducción}

Nuestro equipo desarrolló el proyecto “Autogestión SENA”, una aplicación completa que incluye versión web y móvil, con el objetivo principal de simplificar y agilizar el proceso de asignación de instructores en los centros de formación del SENA. Antes, este trámite se hacía con correos electrónicos, hojas de cálculo y muchas llamadas, lo que generaba errores, retrasos y bastante estrés tanto para coordinadores como para los mismos instructores.

El desarrollo de la aplicación siguió un proceso organizado y por etapas.
Comenzamos por entender y documentar lo que se necesitaba (los requisitos, usando documentos como el SRS y las Historias de Usuario). Luego, diseñamos la estructura de la aplicación.

Para construir la solución completa (la parte interna y la externa), usamos herramientas tecnológicas modernas:
Para la lógica interna (Back-end): Usamos el lenguaje Python con el sistema Django.
Para la aplicación web (Front-end): Creamos la interfaz con React, vite y talwindcss.
Para la aplicación móvil: Usamos .NET MAUI, lo que nos permitió construir una única aplicación que funciona en múltiples plataformas (iOS y Android).
