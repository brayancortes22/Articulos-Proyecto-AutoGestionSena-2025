El SENA es una institución colombiana que forma a miles de personas en oficios técnicos. Tiene muchos centros de formación en diferentes lugares del país. Uno de los trabajos más importantes que hacen es asignar instructores a los estudiantes, pero antes lo hacían con correos, hojas de Excel y llamadas por teléfono, lo que causaba muchos errores y demoras.

Por eso reestructuramos el proyecto "Autogestión SENA", una aplicación que automatiza y mejora este proceso. La app web permite a los coordinadores asignar instructores de forma fácil, a los instructores ver sus horarios y tareas, y a los aprendices saber quién es su instructor. Tiene versión web y móvil, con diferentes permisos para cada tipo de usuario.

Reestructuramos el proyecto en 4 meses usando metodología ágil (Scrum), con 7 ciclos de trabajo cortos que nos permitieron probar y mejorar según la opinión de los usuarios reales. Usamos mucho la inteligencia artificial (sobre todo GitHub Copilot) para acelerar el desarrollo, creando código, arreglando problemas y documentando todo.

Este artículo cuenta toda la experiencia, desde la idea inicial hasta tener el sistema funcionando. Explica cómo la colaboración entre humanos y IA cambió la forma de hacer software. Incluye detalles técnicos, decisiones que tomamos, funciones principales y resultados obtenidos.
\item \textbf{Base de Datos}: MySQL con diseño optimizado para consultas complejas y reporting.
\item \textbf{Despliegue}: Contenedorización con Docker para entornos reproducibles y escalables.
\end{itemize}

\subsection{Rol de la Inteligencia Artificial}

La Inteligencia Artificial se integró como socio colaborativo a lo largo de todo el ciclo de desarrollo, no como reemplazo sino como amplificador de las capacidades humanas. Las aplicaciones principales incluyeron:

\begin{itemize}
\item \textbf{Generación de Código}: Creación automática de esqueletos de modelos, vistas y controladores en Django.
\item \textbf{Diseño de Interfaces}: Sugerencias de layouts y componentes de UI optimizados para usabilidad.
\item \textbf{Depuración Asistida}: Análisis de errores y propuestas de soluciones.
\item \textbf{Documentación}: Generación automática de documentación técnica y guías de usuario.
\item \textbf{Pruebas}: Creación de casos de prueba y escenarios de testing.
\end{itemize}

\subsection{Contribuciones del Artículo}

Este trabajo contribuye al campo de la ingeniería de software educativa de las siguientes maneras:

\begin{enumerate}
\item \textbf{Caso de Estudio Detallado}: Documentación completa de un proyecto real de desarrollo full-stack en contexto colombiano.
\item \textbf{Metodología de Colaboración Humano-IA}: Marco práctico para integrar herramientas de IA en procesos de desarrollo de software educativo.
\item \textbf{Lecciones Aprendidas}: Análisis de desafíos encontrados y soluciones implementadas en entornos de formación técnica.
\item \textbf{Guía Reproducible}: Recursos y procedimientos que permiten replicar el enfoque en otros contextos institucionales.
\end{enumerate}

El resto del artículo se estructura como sigue: la Sección~\ref{sec:literatura} revisa trabajos relacionados; la Sección~\ref{sec:metodologia} detalla la metodología empleada; la Sección~\ref{sec:implementacion} describe la implementación técnica; la Sección~\ref{sec:resultados} presenta los resultados obtenidos; la Sección~\ref{sec:discusion} analiza los hallazgos; finalmente, la Sección~\ref{sec:conclusiones} resume las contribuciones y direcciones futuras.
