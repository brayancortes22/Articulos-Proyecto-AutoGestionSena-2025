\section{Introducción}

Nuestro equipo desarrolló el proyecto “Autogestión SENA”, una aplicación completa que incluye versión web y móvil, con el objetivo central de simplificar y agilizar el proceso de asignación de instructores en los centros de formación del SENA. El proceso tradicional dependía de correos, hojas de cálculo y llamadas telefónicas, lo que causaba errores, pérdida de trazabilidad y retrasos en las asignaciones. Frente a estas ineficiencias, diseñamos una solución que reduce la carga administrativa y mejora la experiencia de instructores y coordinadores.

El desarrollo se llevó a cabo mediante un proceso por etapas: recolección de requisitos (SRS, historias de usuario), prototipado y pruebas de usabilidad, desarrollo del back-end y front-end y despliegue en un entorno reproducible (Docker). Para la implementación adoptamos tecnologías modernas:
\begin{itemize}
	\item Back-end: Python con Django.
	\item Front-end: React con Vite y Tailwind CSS.
	\item Móvil: .NET MAUI para generar una app multiplataforma (Android e iOS).
	\item Base de datos: MySQL.
\end{itemize}

La Inteligencia Artificial (IA) se integró como herramienta de apoyo a lo largo del ciclo de desarrollo: generación de esqueletos y endpoints, sugerencias de diseño y accesibilidad para UI, asistencia en depuración, y apoyo en el diseño de pruebas y documentación.
