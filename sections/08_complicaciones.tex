\section{Complicaciones y mitigaciones}

Durante el desarrollo del proyecto se identificaron varias complicaciones que exigieron atención y mitigación por parte del equipo. En esta sección se detallan las más relevantes y las acciones tomadas.

\subsection{Integración Front-end / Back-end}
\textbf{Problema}: Discrepancias en contratos JSON, problemas CORS y errores de validación durante la integración.
\textbf{Mitigación}: Definición clara de contratos API, tests de integración automáticos y versionamiento de API.

\subsection{Diferencias de entornos}
\textbf{Problema}: Comportamiento distinto entre entornos locales, staging y producción (variables de entorno, dependencias de librerías).
\textbf{Mitigación}: Uso de contenedores Docker, plantillas .env y pruebas de despliegue en entornos de staging reproducibles.

\subsection{Limitaciones de la IA}
\textbf{Problema}: Sugerencias incompletas, supuestos incorrectos o falta de contexto en prompts.
\textbf{Mitigación}: Revisión humana de propuestas, anotación explícita de supuestos y creación de tests para validar outputs de IA.

\subsection{Seguridad y exposición de credenciales}
\textbf{Problema}: Riesgo de exponer secretos en scripts generados o en logs.
\textbf{Mitigación}: Uso de almacenamiento seguro de secretos (p. ej., Azure Key Vault), auditoría de scripts y políticas para evitar la inclusión de credenciales en el repositorio.

\subsection{Problemas de UX y accesibilidad}
\textbf{Problema}: Propuestas de UI generadas por IA que no consideraban accesibilidad o la realidad del usuario final.
\textbf{Mitigación}: Validación con usuarios reales e instructores, ajustes de diseño y aplicación de estándares de accesibilidad.
