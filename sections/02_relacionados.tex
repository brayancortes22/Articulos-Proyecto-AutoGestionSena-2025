\section{Marco teórico y trabajos relacionados}
La integración de Inteligencia Artificial en la enseñanza de la programación ha emergido como un área de investigación relevante en años recientes. Cadena y Juárez \cite{cadena2023ensenanza} demuestran que los agentes conversacionales basados en IA pueden actuar como asistentes educativos efectivos, proporcionando soporte continuo a estudiantes de programación.

Terán \cite{teran2023implementacion} examina las implicaciones éticas del uso de ChatGPT en el aula, estableciendo que la IA debe complementar el aprendizaje humano sin reemplazar el pensamiento crítico. Este enfoque ético es fundamental para garantizar que los estudiantes desarrollen competencias genuinas.

Pincay \cite{pincay2025aplicaciones} identifica aplicaciones específicas de la IA en el campo de la programación, destacando su utilidad en la generación de código, debugging y explicación de conceptos técnicos complejos. Estos estudios establecen el marco teórico para comprender cómo la IA puede acelerar el desarrollo de software mientras mantiene el control humano sobre decisiones arquitectónicas y de diseño.

Sin embargo, se identifican vacíos en estudios que documenten experiencias completas de desarrollo full-stack asistido por IA en contextos educativos latinoamericanos, particularmente en instituciones como el SENA.
