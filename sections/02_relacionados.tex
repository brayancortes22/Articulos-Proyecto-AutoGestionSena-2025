\section{Marco teórico y trabajos relacionados}
La integración de inteligencia artificial en la enseñanza de la programación ha sido objeto de estudio reciente. Diversos trabajos han mostrado que los agentes conversacionales y asistentes de código pueden apoyar el aprendizaje y la resolución de dudas en tiempo real, aportando beneficios pedagógicos y prácticos.

Al mismo tiempo, la literatura advierte sobre consideraciones éticas y pedagógicas: la IA debe complementar el aprendizaje humano sin sustituir el pensamiento crítico ni las competencias fundamentales. Estudios sobre aplicaciones concretas de IA en programación destacan su utilidad en generación de código, debugging y explicación de conceptos técnicos.

No obstante, existen pocos estudios que documenten experiencias completas de desarrollo full-stack asistido por IA en contextos educativos latinoamericanos. El presente trabajo aporta una experiencia práctica aplicada al SENA que ayuda a cubrir ese vacío.
