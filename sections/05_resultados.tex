\section{Gráfico}
\begin{center}
	\includegraphics[width=0.9\textwidth]{graphics/ciclo_desarrollo_ia.pdf}
\end{center}
\vspace{1em}
\noindent\textbf{Figura 1:} Ciclo de desarrollo asistido por IA (caso de estudio “Autogestión SENA”).
\label{fig:ciclo_desarrollo_ia}
\vspace{1em}

\subsection{Métricas y visualizaciones}
\begin{figure}[htbp]
	\centering
	\includegraphics[width=0.48\textwidth]{graphics/uso_ia_barras.png}
	\includegraphics[width=0.48\textwidth]{graphics/distribucion_tareas.png}
	\caption{Izquierda: Uso de IA por tipo de tarea (número de prompts/consultas). Derecha: Distribución del esfuerzo por tipo de tarea durante el proyecto.}
	\label{fig:uso_ia_distribucion}
\end{figure}

\vspace{1em}
\begin{table}[htbp]
\centering
\caption{Comparación de Frameworks de Desarrollo Web}
\label{tab:frameworks}
\begin{tabular}{lccccc}
\toprule
\textbf{Framework} & \textbf{Lenguaje} & \textbf{Performance} & \textbf{Curva Aprendizaje} & \textbf{Comunidad} & \textbf{Puntuación} \\
\midrule
React & JavaScript & Alta & Media & Excelente & 9.2 \\
Angular & TypeScript & Alta & Alta & Excelente & 8.7 \\
Vue.js & JavaScript & Alta & Baja & Buena & 8.9 \\
Django & Python & Media & Media & Excelente & 8.5 \\
Spring Boot & Java & Alta & Alta & Excelente & 8.8 \\
Laravel & PHP & Media & Baja & Buena & 8.1 \\
Express.js & JavaScript & Alta & Baja & Buena & 8.3 \\
\bottomrule
\end{tabular}
\end{table}

\begin{figure}[htbp]
	\centering
	\includegraphics[width=0.48\textwidth]{graphics/sprint_timeline.png}
	\includegraphics[width=0.48\textwidth]{graphics/complicaciones_reportadas.png}
	\caption{Cronograma de sprints (izquierda) y número de incidencias reportadas por categoría (derecha).}
	\label{fig:sprints_complicaciones}
\end{figure}

\begin{figure}[htbp]
	\centering
	\includegraphics[width=0.7\textwidth]{graphics/comparacion_metodologias.png}
	\caption{Comparación de metodologías ágiles en factores como flexibilidad, velocidad de entrega y calidad de código.}
	\label{fig:comparacion_metodologias}
\end{figure}

\subsection{Resumen de resultados}
El uso de IA contribuyó a acelerar la construcción de esqueletos y endpoints, reducir tiempos de bloqueo en debugging y mejorar la documentación base. Las figuras muestran que el uso de IA fue predominante en generación de código y explicación de funciones, mientras que la distribución de esfuerzo refleja una carga mayor en la implementación y pruebas.

Algunas métricas cualitativas y ejemplos de mejora observada:
\begin{itemize}
	\item Aceleración de tareas repetitivas (plantillas, validaciones) en fases tempranas.
	\item Reducción del tiempo de bloqueo por errores gracias a sugerencias de debugging.
	\item Mejora en la documentación y ejemplos de pruebas generados inicialmente por la IA.
	\item Recomendaciones de UI y accesibilidad que mejoraron la experiencia de usuario.
\end{itemize}

\vspace{3em}

