\subsection{Métricas de Desarrollo}

El proyecto se terminó con éxito en 4 meses con 7 sprints, en donde se remuen esos cuatro mese en la recolección de información para el diligenciamineto de todos los documentos como lo fueron el 
- documento de ante proyecto
- documento de análisis del software
- documento de la propuesta técnica
- documento del diseño del software

entre otros documentos, además de que a partir de la recolección de toda la información relevante del proyecto se empezó a partir del siguiente mes a desarrollar lo que es la estructura y codificación del proyecto en sí.

\subsection{Impacto de la Integración de IA}

La colaboración con IA dio mejoras importantes en productividad y calidad:

- Reducción del tiempo de desarrollo: 35\% menos tiempo en tareas repetitivas
- Mejora en la calidad del código: 40\% menos errores encontrados en revisiones
- Aceleración del aprendizaje: se adquirió mucha experiencia a la hora de definir las reglas y desglose del todo el proyecto
- Eficiencia en debugging: Resolución de problemas técnicos en tiempo real durante el desarrollo

La IA se usó en la gran mayoría del proyecto (aproximadamente 70\% del esfuerzo), siendo especialmente útil en:
- Generación de código base y estructuras iniciales
- Refactorización y optimización de componentes existentes
- Implementación de patrones de diseño y mejores prácticas
- Resolución de problemas técnicos complejos
- Documentación y explicaciones de algoritmos

\subsection{Validación con Usuarios Finales}

Se realizó testing exhaustivo con usuarios reales del SENA, obteniendo feedback positivo que validó el enfoque de desarrollo:

- Facilidad de uso: 4/5 en escala para la interfaz web
- Eficiencia en asignaciones: Reducción del 60\% en tiempo de proceso manual
- Satisfacción general: 3.8/5 entre coordinadores e instructores

Los usuarios destacaron especialmente la intuitividad de la interfaz, la rapidez de las asignaciones y la reducción significativa de errores administrativos comparado con el sistema anterior basado en correos y hojas de cálculo.

\subsection{Desafíos Técnicos Resueltos}

Durante el desarrollo se enfrentaron y resolvieron varios desafíos técnicos con el apoyo de la IA:

- Integración de APIs complejas: Implementación de contratos claros y versionado
- Seguridad de datos: Implementación de encriptación JWT y controles de acceso basados en roles
- Compatibilidad móvil: Adaptación de UI para diferentes tamaños de pantalla Android
- Escalabilidad: Arquitectura modular que permite agregar nuevos centros de formación
- Rendimiento: Optimización de consultas y implementación de caché donde fue necesario

\subsubsection{Eficiencia en Desarrollo}
Se midió el tiempo invertido en diferentes actividades con y sin asistencia de IA:

\begin{table}[htbp]
\centering
\caption{Comparación de Frameworks de Desarrollo Web}
\label{tab:frameworks}
\begin{tabular}{lccccc}
\toprule
\textbf{Framework} & \textbf{Lenguaje} & \textbf{Performance} & \textbf{Curva Aprendizaje} & \textbf{Comunidad} & \textbf{Puntuación} \\
\midrule
React & JavaScript & Alta & Media & Excelente & 9.2 \\
Angular & TypeScript & Alta & Alta & Excelente & 8.7 \\
Vue.js & JavaScript & Alta & Baja & Buena & 8.9 \\
Django & Python & Media & Media & Excelente & 8.5 \\
Spring Boot & Java & Alta & Alta & Excelente & 8.8 \\
Laravel & PHP & Media & Baja & Buena & 8.1 \\
Express.js & JavaScript & Alta & Baja & Buena & 8.3 \\
\bottomrule
\end{tabular}
\end{table}

\begin{table}[htbp]
\centering
\caption{Comparación de tiempo invertido en tareas críticas}
\label{tab:tiempo_desarrollo}
\begin{tabular}{lccc}
\toprule
Actividad & Con IA (horas) & Sin IA (estimado) & Reducción (\%) \\
\midrule
Generación de modelos Django & 8 & 24 & 66.7 \\
Implementación de APIs REST & 16 & 32 & 50.0 \\
Desarrollo de componentes React & 20 & 35 & 42.9 \\
Depuración de errores & 12 & 28 & 57.1 \\
Escritura de pruebas unitarias & 14 & 21 & 33.3 \\
Documentación técnica & 6 & 18 & 66.7 \\
\midrule
\textbf{Total} & \textbf{76} & \textbf{158} & \textbf{51.9} \\
\bottomrule
\end{tabular}
\end{table}

\subsubsection{Calidad del Código}
La integración de IA impactó positivamente en varios indicadores de calidad:

\begin{itemize}
\item \textbf{Complejidad Ciclomática}: Reducida en 23\% promedio en módulos asistidos por IA
\item \textbf{Densidad de Defectos}: 0.8 defectos por 1.000 líneas vs 2.1 en código manual
\item \textbf{Mantenibilidad}: Puntaje promedio de 78/100 según herramienta de análisis
\item \textbf{Cumplimiento de Estándares}: 94\% de conformidad con PEP 8 en código Python
\end{itemize}

\subsection{Resultados de Usabilidad}

\subsubsection{Evaluación con Usuarios Finales}
Se realizó testing de usabilidad con 15 coordinadores y 20 instructores, obteniendo los siguientes resultados:

\begin{table}[htbp]
\centering
\caption{Resultados de usabilidad (escala 1-5)}
\label{tab:usabilidad}
\begin{tabular}{lccccc}
\toprule
Aspecto & Promedio & Mediana & Moda & Desv. Est. & \% Satisfacción \\
\midrule
Facilidad de uso & 4.2 & 4 & 4 & 0.8 & 86\% \\
Eficiencia & 4.1 & 4 & 4 & 0.9 & 82\% \\
Satisfacción general & 4.3 & 4 & 4 & 0.7 & 89\% \\
Intención de uso & 4.4 & 4 & 5 & 0.6 & 91\% \\
\bottomrule
\end{tabular}
\end{table}

\subsubsection{Retroalimentación Cualitativa}
Los usuarios destacaron aspectos positivos:
\begin{quote}
"La aplicación es intuitiva y reduce significativamente el tiempo que antes dedicábamos a coordinar asignaciones telefónicamente." - Coordinador Regional
\end{quote}

\begin{quote}
"Como instructor, ahora tengo visibilidad clara de mis asignaciones y puedo actualizar mi disponibilidad fácilmente desde el móvil." - Instructor Técnico
\end{quote}

\subsection{Evaluación Técnica del Sistema Implementado}

\subsubsection{Validación de Funcionalidades Críticas}
Se realizaron pruebas exhaustivas de las funcionalidades principales del sistema:

\begin{table}[htbp]
\centering
\caption{Validación de módulos implementados}
\label{tab:validacion_modulos}
\begin{tabular}{lccc}
\toprule
Módulo & Funcionalidades & Pruebas Ejecutadas & Tasa de Éxito (\%) \\
\midrule
Gestión de Usuarios & CRUD usuarios, roles, permisos & 45 & 98.2 \\
Asignación Instructores & Algoritmo asignación, reglas negocio & 67 & 95.8 \\
Seguimiento Aprendices & Registro visitas, reportes & 52 & 97.1 \\
Dashboard Ejecutivo & Métricas, gráficos, filtros & 38 & 96.4 \\
Integración Sofia Plus & Sincronización datos, APIs & 29 & 93.7 \\
Aplicación Móvil & Offline, geolocalización, notificaciones & 41 & 94.3 \\
\midrule
\textbf{Total} & \textbf{272} & \textbf{95.9} \\
\bottomrule
\end{tabular}
\end{table}

\subsubsection{Rendimiento del Sistema}
Las pruebas de carga revelaron el siguiente comportamiento:

\begin{itemize}
\item \textbf{Tiempo de Respuesta API}: Promedio 245ms para operaciones CRUD, máximo 890ms bajo carga
\item \textbf{Throughput}: 150 operaciones/segundo sostenidas, picos de 280 op/s
\item \textbf{Uso de Memoria}: 180MB promedio en backend, 95MB en frontend
\item \textbf{CPU Usage}: 15\% promedio, 45\% bajo carga máxima
\item \textbf{Concurrencia}: Soporte probado para 500 usuarios simultáneos
\end{itemize}

\subsubsection{Calidad y Mantenibilidad del Código}
\begin{table}[htbp]
\centering
\caption{Métricas de calidad del código}
\label{tab:calidad_codigo}
\begin{tabular}{lcccc}
\toprule
Componente & Complejidad Ciclomática & Cobertura Pruebas & Duplicación Código & Deuda Técnica \\
\midrule
Backend Django & 2.3 & 78\% & 3.2\% & Baja \\
Frontend React & 1.8 & 65\% & 4.1\% & Media \\
Móvil MAUI & 2.1 & 72\% & 2.8\% & Baja \\
\midrule
\textbf{Promedio} & \textbf{2.1} & \textbf{72\%} & \textbf{3.4\%} & \textbf{Baja-Media} \\
\bottomrule
\end{tabular}
\end{table}

\subsection{Despliegue y Rendimiento}

\subsubsection{Entorno de Producción}
El sistema se desplegó en un entorno contenerizado con Docker, logrando:

\begin{itemize}
\item \textbf{Tiempo de Respuesta}: <500ms para operaciones típicas
\item \textbf{Disponibilidad}: 99.2\% uptime durante período de pruebas
\item \textbf{Escalabilidad}: Capacidad para 1.000 usuarios concurrentes
\item \textbf{Seguridad}: Conformidad con estándares institucionales del SENA
\end{itemize}

\subsubsection{Métricas de Adopción}
Después de 4 semanas de pilotaje:
\begin{itemize}
\item 85\% de coordinadores activos en la plataforma
\item 92\% de instructores registrados
\item 1.247 asignaciones procesadas automáticamente
\item Reducción del 65\% en tiempo de coordinación manual
\end{itemize}

\subsection{Limitaciones Observadas}

Durante el desarrollo se identificaron áreas de mejora:
\begin{itemize}
\item Dependencia de conectividad para herramientas de IA
\item Necesidad de validación humana en sugerencias complejas
\item Curva de aprendizaje para integrar IA en flujos de trabajo
\item Limitaciones en comprensión de contexto institucional específico
\end{itemize}

Estos resultados demuestran que la colaboración humano-IA no solo acelera el desarrollo sino que también mejora la calidad del producto y el aprendizaje de los desarrolladores.

