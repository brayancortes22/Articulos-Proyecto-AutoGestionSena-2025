\section{Resultados}
En el desarrollo de "Autogestión SENA" se observaron resultados cuantitativos y cualitativos derivados del uso de IA como herramienta de producción y aprendizaje.

\subsection{Impacto en la productividad}
Los asistentes de código e IA ayudaron a reducir el tiempo dedicado a tareas repetitivas (esqueletos de endpoints, plantillas UI, pruebas iniciales), lo que liberó tiempo para la mejora de la experiencia de usuario y el diseño del dominio. La observación práctica durante los 7 sprints mostró una disminución de bloqueos técnicos y una mayor velocidad de entrega.

\subsection{Métricas de calidad y despliegue}
Se analizaron métricas tipo DORA (frecuencia de despliegue y lead time) y tendencias de calidad durante el periodo de trabajo. Los resultados indicaron mejoras en tiempos de entrega y reducción de errores en producción, medidos a lo largo de los sprints.

\subsection{Distribución de esfuerzo}
La Figura~\ref{fig:distribucion_tareas} muestra un ejemplo de la distribución del esfuerzo por tipo de tarea, reflejando que la IA contribuyó principalmente en generación de código y apoyo en pruebas.

\begin{figure}[htbp]
  \centering
  \includegraphics[width=0.65\linewidth]{graphics/distribucion_tareas.pdf}
  \caption{Distribución aproximada del esfuerzo por tipo de tarea.}
  \label{fig:distribucion_tareas}
\end{figure}

\subsection{Reducción del tiempo de escritura de código}
La IA abordó tareas repetitivas y generó fragmentos de código (esqueletos de endpoints, validaciones, componentes visuales), lo que liberó tiempo para que el equipo se concentrara en la lógica de negocio, la experiencia de usuario y la organización de la información.

\subsection{Síntesis}
La IA no sustituyó las decisiones humanas; en cambio, permitió que el equipo dedicara más tiempo a decisiones críticas de UX, arquitectura y priorización. En conjunto, el soporte de IA contribuyó a un desarrollo más rápido y con mayor foco en los aspectos cualitativos del proyecto.

\subsection{Reflexión}
Al crear la app “Autogestión SENA” vimos claramente lo útil que es la Inteligencia Artificial. La IA nos ayudó mucho como herramienta y como compañero de aprendizaje. Tuvo dos beneficios principales:
\begin{itemize}
  \item Resolver dudas rápido
  \item Hacer el trabajo más rápido
\end{itemize}
El mayor problema en los proyectos en equipo es quedarse atascado esperando que alguien te ayude con una duda técnica. Con la IA eso se redujo: teníamos un "experto" disponible 24 horas. Cuando teníamos un bug o una función difícil de escribir, la IA proporcionó sugerencias y explicaciones que se aplicaron tras revisión humana.
\section{Resultados}
En el desarrollo de "Autogestión SENA" se observaron resultados cuantitativos y cualitativos derivados del uso de IA como herramienta de producción y aprendizaje.

\subsection{Impacto en la productividad}
Los asistentes de código e IA ayudaron a reducir el tiempo dedicado a tareas repetitivas (esqueletos de endpoints, plantillas UI, pruebas iniciales), lo que liberó tiempo para la mejora de la experiencia de usuario y el diseño del dominio. La observación práctica durante los 7 sprints mostró una disminución de bloqueos técnicos y una mayor velocidad de entrega.

\subsection{Métricas de calidad y despliegue}
Se analizaron métricas tipo DORA (frecuencia de despliegue y lead time) y tendencias de calidad durante el periodo de trabajo. Los resultados indicaron mejora en tiempos de entrega y reducción de errores en producción, medidos a lo largo de los sprints. A continuación se muestra un desglose del esfuerzo por tipo de tarea.

\subsection{Distribución de esfuerzo}
La Figura~\ref{fig:distribucion_tareas} muestra un ejemplo de la distribución del esfuerzo por tipo de tarea, reflejando que la IA contribuyó principalmente en generación de código y apoyo en pruebas.

\begin{figure}[htbp]
  \centering
  \includegraphics[width=0.65\linewidth]{graphics/distribucion_tareas.pdf}
  \caption{Distribución aproximada del esfuerzo por tipo de tarea.}
  \label{fig:distribucion_tareas}
\end{figure}

\subsection{Síntesis}
La IA no sustituyó las decisiones humanas; en cambio, permitió que el equipo dedicara más tiempo a decisiones críticas de UX, arquitectura y priorización. En conjunto, el soporte de IA contribuyó a un desarrollo más rápido y con mayor foco en los aspectos cualitativos del proyecto.
\section{Resultados}
Reflexión
Al crear la app “Autogestión SENA” vimos claramente lo útil que es la Inteligencia Artificial. La IA nos ayudó mucho como herramienta y como compañero de aprendizaje. Tuvo dos grandes beneficios para el equipo:

Resolver dudas rápido
Hacer el trabajo más rápido

Resolver dudas y bloqueos al instante
El mayor problema en los proyectos en equipo es quedarse atascado esperando que alguien te ayude con una duda técnica. Con la IA eso desapareció: teníamos un “experto” disponible 24 horas.

Cuando teníamos un bug, una función difícil de escribir o queríamos un código de calidad, se lo preguntábamos a la IA. Ella nos daba la solución al instante y nos explicaba cómo funcionaba y por qué era mejor hacerlo de esa manera.

La IA creó rápido cosas como:
- Estructuras básicas de las páginas web (endpoints en Django)
- Animaciones bonitas en la parte visible (front-end)
- Código repetitivo que normalmente toma mucho tiempo

Gracias a eso escribimos el código mucho más rápido. Tuvimos más tiempo para lo realmente importante: hablar de cómo hacer la app fácil de usar, cómo organizar bien la información y cómo mejorar la experiencia del usuario. Esas cosas sí las tiene que hacer una persona, y con la IA pudimos dedicarles más tiempo y energía.

En resumen: la IA no reemplazó al equipo, sino que nos quitó el trabajo aburrido y nos dejó enfocarnos en lo que de verdad importa.
La disponibilidad de asistencia basada en IA redujo significativamente los tiempos de espera por ayuda técnica. Ante errores o dudas, el equipo obtuvo respuestas rápidas que permitieron mantener el flujo de trabajo.

\subsection{Reducción del tiempo de escritura de código}
La IA abordó tareas repetitivas y generó fragmentos de código (esqueletos de endpoints, validaciones, componentes visuales), lo que liberó tiempo para que el equipo se concentrara en la lógica de negocio, la experiencia de usuario y la organización de la información.

\subsection{Síntesis}
En conjunto, la IA no reemplazó al equipo humano; más bien permitió eliminar trabajo mecánico y dedicar más esfuerzo a tareas que requieren juicio, creatividad y diseño.