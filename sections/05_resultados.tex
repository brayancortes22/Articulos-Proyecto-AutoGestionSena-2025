\section{Métricas de Desarrollo}

El proyecto se terminó con éxito en 4 meses con 7 sprints, en donde se remue sos cuatro mese en la recolección de información para el diligenciamineto de todos los documentos como lo fueron el 
\begin{itemize}
\item documento de ante proyecto
\item documento de análisis del software
\item documento de la propuesta técnica
\item documento del diseño del software
\end{itemize}

entre otros documentos, además de que a partir de la recolección de toda la información relevante del proyecto se empezó a partir del siguiente mes a desarrollar lo que es la estructura y codificación del proyecto en sí.

\section{Impacto de la Integración de IA}

La colaboración con IA dio mejoras importantes en productividad y calidad:

\begin{itemize}
\item Reducción del tiempo de desarrollo: 35\% menos tiempo en tareas repetitivas
\item Mejora en la calidad del código: 40\% menos errores encontrados en revisiones
\item Aceleración del aprendizaje: se adquirió mucha experiencia a la hora de definir las reglas y desglose del todo el proyecto
\item Eficiencia en debugging: Resolución de problemas técnicos en tiempo real durante el desarrollo
\end{itemize}

La IA se usó en la gran mayoría del proyecto (aproximadamente 70\% del esfuerzo), siendo especialmente útil en:
\begin{itemize}
\item Generación de código base y estructuras iniciales
\item Refactorización y optimización de componentes existentes
\item Implementación de patrones de diseño y mejores prácticas
\item Resolución de problemas técnicos complejos
\item Documentación y explicaciones de algoritmos
\end{itemize}

\section{Validación con Usuarios Finales}

Se realizó testing exhaustivo con usuarios reales del SENA, obteniendo feedback positivo que validó el enfoque de desarrollo:

\begin{itemize}
\item Facilidad de uso: 4/5 en escala para la interfaz web
\item Eficiencia en asignaciones: Reducción del 60\% en tiempo de proceso manual
\item Satisfacción general: 3.8/5 entre coordinadores e instructores
\end{itemize}

Los usuarios destacaron especialmente la intuitividad de la interfaz, la rapidez de las asignaciones y la reducción significativa de errores administrativos comparado con el sistema anterior basado en correos y hojas de cálculo.

\section{Desafíos Técnicos Resueltos}

Durante el desarrollo se enfrentaron y resolvieron varios desafíos técnicos con el apoyo de la IA:

\begin{itemize}
\item Integración de APIs complejas: Implementación de contratos claros y versionado
\item Seguridad de datos: Implementación de encriptación JWT y controles de acceso basados en roles
\item Compatibilidad móvil: Adaptación de UI para diferentes tamaños de pantalla Android
\item Escalabilidad: Arquitectura modular que permite agregar nuevos centros de formación
\item Rendimiento: Optimización de consultas y implementación de caché donde fue necesario
\end{itemize}