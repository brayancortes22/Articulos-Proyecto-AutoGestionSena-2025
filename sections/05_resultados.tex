\section{Resultados}
Al crear la aplicación ``Autogestión SENA'' se evidenció claramente la utilidad de la Inteligencia Artificial. La IA proporcionó dos grandes beneficios fundamentales para el equipo:

\subsection{Resolución de dudas y bloqueos al instante}
El mayor problema en los proyectos en equipo es quedarse atascado esperando que alguien ayude con una duda técnica. Con la IA esto desapareció: el equipo contaba con un ``experto'' disponible 24 horas.

Cuando había un error, una función difícil o se necesitaba escribir código limpio y correcto, simplemente se consultaba a la IA. En segundos proporcionaba la respuesta o el código necesario. Así nadie se quedaba parado esperando y todos podían seguir trabajando sin interrupciones.

\subsection{Reducción del tiempo de escritura de código}
La parte creativa y la lógica del proyecto (qué debía hacer la aplicación) la realizaron las personas, porque eso requiere pensamiento humano. Sin embargo, la IA ayudó significativamente con las partes más mecánicas y repetitivas.

La IA generó rápidamente componentes como:
\begin{itemize}
    \item Estructuras básicas de las páginas web (endpoints en Django)
    \item Animaciones atractivas en la parte visible (front-end)
    \item Código repetitivo que normalmente consume mucho tiempo
\end{itemize}

Gracias a esto, el código se escribió mucho más rápido. El equipo tuvo más tiempo para lo realmente importante: discutir cómo hacer la aplicación fácil de usar, cómo organizar bien la información y cómo mejorar la experiencia del usuario. Estas tareas requieren intervención humana, y con la IA se pudo dedicar más tiempo y energía a ellas.

\subsection{Síntesis de resultados}
En resumen: la IA no reemplazó al equipo, sino que eliminó el trabajo repetitivo y permitió enfocarse en lo que realmente importa. La integración de herramientas de IA en el proceso de desarrollo \cite{cadena2023ensenanza} demostró ser una estrategia efectiva para acelerar el aprendizaje y mejorar la productividad del equipo \cite{teran2023implementacion}.