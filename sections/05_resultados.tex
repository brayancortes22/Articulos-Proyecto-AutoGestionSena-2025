\section{Métricas de Desarrollo}

El proyecto se terminó con éxito en 4 meses con 7 sprints, en donde se remue sos cuatro mese en la recolección de información para el diligenciamineto de todos los documentos como lo fueron el
\begin{itemize}
\item documento de ante proyecto
\item documento de análisis del software
\item documento de la propuesta técnica
\item documento del diseño del software
\end{itemize}

entre otros documentos, además de que a partir de la recolección de toda la información relevante del proyecto se empezó a partir del siguiente mes a desarrollar lo que es la estructura y codificación del proyecto en sí.

\subsection{Cronograma de Ejecución}

El desarrollo se estructuró en siete sprints de dos semanas cada uno, con hitos específicos:

\textbf{Sprint 1 - Análisis y Diseño:}
\begin{itemize}
\item Recopilación completa de requisitos funcionales y no funcionales
\item Diseño de arquitectura técnica y modelo de datos
\item Definición de interfaces de usuario y flujos de navegación
\item Configuración inicial de entornos de desarrollo
\end{itemize}

\textbf{Sprint 2 - Backend Core:}
\begin{itemize}
\item Implementación de modelos de datos y migraciones
\item Desarrollo de APIs REST básicas de autenticación
\item Configuración de base de datos y sistema de cache
\item Implementación de algoritmos de asignación automática
\end{itemize}

\textbf{Sprint 3 - Frontend Web:}
\begin{itemize}
\item Desarrollo de componentes de interfaz de usuario
\item Implementación de formularios y validaciones
\item Integración con APIs del backend
\item Optimización de rendimiento y responsividad
\end{itemize}

\textbf{Sprint 4 - Aplicación Móvil:}
\begin{itemize}
\item Desarrollo de interfaces móviles nativas
\item Implementación de consumo de APIs REST
\item Configuración de notificaciones push
\item Testing en dispositivos físicos
\end{itemize}

\textbf{Sprint 5 - Integración y Testing:}
\begin{itemize}
\item Integración de todos los componentes del sistema
\item Desarrollo de pruebas automatizadas
\item Optimización de rendimiento del sistema
\item Validación de seguridad y compatibilidad
\end{itemize}

\textbf{Sprint 6 - Despliegue y Documentación:}
\begin{itemize}
\item Configuración de entornos de producción
\item Desarrollo de scripts de despliegue automatizado
\item Documentación técnica completa del sistema
\item Capacitación de usuarios finales
\end{itemize}

\textbf{Sprint 7 - Optimización y Mantenimiento:}
\begin{itemize}
\item Optimización de consultas de base de datos
\item Mejora de la experiencia de usuario
\item Implementación de monitoreo y logging
\item Preparación para mantenimiento a largo plazo
\end{itemize}

\section{Impacto de la Integración de IA}

La colaboración con IA dio mejoras importantes en productividad y calidad:

\begin{itemize}
\item Reducción del tiempo de desarrollo: 35\% menos tiempo en tareas repetitivas
\item Mejora en la calidad del código: 40\% menos errores encontrados en revisiones
\item Aceleración del aprendizaje: se adquirió mucha experiencia a la hora de definir las reglas y desglose del todo el proyecto
\item Eficiencia en debugging: Resolución de problemas técnicos en tiempo real durante el desarrollo
\end{itemize}

La IA se usó en la gran mayoría del proyecto (aproximadamente 70\% del esfuerzo), siendo especialmente útil en:
\begin{itemize}
\item Generación de código base y estructuras iniciales
\item Refactorización y optimización de componentes existentes
\item Implementación de patrones de diseño y mejores prácticas
\item Resolución de problemas técnicos complejos
\item Documentación y explicaciones de algoritmos
\end{itemize}

\subsection{Métricas Cuantitativas de Rendimiento}

Para medir el éxito del proyecto, se definieron métricas específicas de rendimiento del sistema:

\textbf{Métricas de Usabilidad:}
\begin{itemize}
\item Tiempo de carga promedio de páginas: < 2 segundos
\item Tasa de éxito en operaciones críticas: > 95\%
\item Número de clics necesarios para completar tareas: reducido en 30\%
\item Satisfacción del usuario medida por encuestas: 4.2/5 puntos
\end{itemize}

\textbf{Métricas de Rendimiento Técnico:}
\begin{itemize}
\item Tiempo de respuesta de APIs: < 500ms para operaciones estándar
\item Uso de CPU en servidores: < 60\% bajo carga normal
\item Uso de memoria: < 70\% de capacidad disponible
\item Disponibilidad del sistema: 99.5\% uptime mensual
\end{itemize}

\textbf{Métricas de Escalabilidad:}
\begin{itemize}
\item Capacidad de usuarios concurrentes: 1000+ usuarios simultáneos
\item Tiempo de respaldo de base de datos: < 30 minutos
\item Capacidad de almacenamiento: escalable según necesidades
\item Soporte para múltiples instituciones educativas
\end{itemize}

\begin{figure}[htbp]
\centering
\includegraphics[width=\columnwidth]{graphics/distribucion_tareas.pdf}
\caption{Distribución del esfuerzo por tipo de tarea}
\label{fig:distribucion_tareas}
\end{figure}

\begin{figure}[htbp]
\centering
\includegraphics[width=\columnwidth]{graphics/uso_ia_barras.pdf}
\caption{Uso de IA por tipo de tarea}
\label{fig:uso_ia}
\end{figure}

\begin{figure}[htbp]
\centering
\includegraphics[width=\columnwidth]{graphics/metricas_rendimiento.pdf}
\caption{Métricas de Rendimiento del Sistema de Autogestión}
\label{fig:metricas_rendimiento}
\end{figure}

\subsection{Análisis Comparativo de Tecnologías}

El proyecto implementó una comparación detallada entre diferentes tecnologías y frameworks utilizados, evaluando factores como:

\textbf{Criterios de Evaluación:}
\begin{itemize}
\item Facilidad de desarrollo y curva de aprendizaje
\item Rendimiento y escalabilidad
\item Mantenibilidad y extensibilidad del código
\item Compatibilidad con requisitos del SENA
\item Comunidad de soporte y documentación disponible
\end{itemize}

\textbf{Resultados de la Evaluación:}
\begin{itemize}
\item \textbf{Django vs. otros frameworks Python:} Mejor elección por madurez, documentación y ecosistema
\item \textbf{React vs. Vue/Angular:} Mayor flexibilidad y mejor integración con TypeScript
\item \textbf{.NET MAUI vs. Flutter/React Native:} Mejor integración con ecosistema Microsoft y Windows
\item \textbf{MySQL vs. PostgreSQL:} Suficiente para requisitos actuales, con posibilidad de migración futura
\end{itemize}

Esta evaluación sistemática permitió tomar decisiones técnicas fundamentadas que optimizaron el desarrollo y aseguraron la viabilidad a largo plazo del proyecto.

\section{Validación con Usuarios Finales}

Se realizó testing exhaustivo con usuarios reales del SENA, obteniendo feedback positivo que validó el enfoque de desarrollo:

\begin{itemize}
\item Facilidad de uso: 4/5 en escala para la interfaz web
\item Eficiencia en asignaciones: Reducción del 60\% en tiempo de proceso manual
\item Satisfacción general: 3.8/5 entre coordinadores e instructores
\end{itemize}

Los usuarios destacaron especialmente la intuitividad de la interfaz, la rapidez de las asignaciones y la reducción significativa de errores administrativos comparado con el sistema anterior basado en correos y hojas de cálculo.

\section{Desafíos Técnicos Resueltos}

Durante el desarrollo se enfrentaron y resolvieron varios desafíos técnicos con el apoyo de la IA:

\begin{itemize}
\item Integración de APIs complejas: Implementación de contratos claros y versionado
\item Seguridad de datos: Implementación de encriptación JWT y controles de acceso basados en roles
\item Compatibilidad móvil: Adaptación de UI para diferentes tamaños de pantalla Android
\item Escalabilidad: Arquitectura modular que permite agregar nuevos centros de formación
\item Rendimiento: Optimización de consultas y implementación de caché donde fue necesario
\end{itemize}