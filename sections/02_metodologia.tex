El proyecto se reestructuró siguiendo una metodología ágil llamada Scrum, durante 4 meses divididos en 7 sprints (ciclos de trabajo). Esta forma de trabajar nos dio flexibilidad para escuchar a los usuarios reales (instructores y coordinadores del SENA) y cambiar el proyecto según sus necesidades.

\section{Análisis de Requisitos y Diseño del Sistema}

Empezamos recopilando todo lo que el sistema debía hacer, identificando cinco tipos de usuarios: coordinadores, instructores, Aprendices, administradores y operadores de Sofia plus. Las funciones incluyeron manejar usuarios con seguridad, asignar instructores automática o manualmente, ver horarios, hacer reportes y tener interfaces que funcionen en cualquier dispositivo. La IA ayudó mucho aquí, creando diagramas complejos de relaciones entre datos, flujos de navegación y sugerencias de diseño que aceleraron el trabajo inicial.

El modelo de datos se organizó alrededor de elementos clave como usuarios, roles, Aprendices, instructores, asignaciones y seguimientos, con diferentes tipos de conexiones entre ellos. La IA ayudó a organizar bien la base de datos MySQL, sugiriendo formas de guardar los datos de manera eficiente y segura.

\section{Arquitectura Técnica}

El sistema se rediseñó siguiendo reglas de organización clara y capacidad de crecer, usando un patrón de capas basado en módulos que hace fácil mantener y actualizar el software. Tomamos la estructura existente y la mejoramos completamente con tecnologías más modernas y mejores prácticas.

Parte back-end (API REST con Django/Python):
\begin{itemize}
\item Django 4.2 con Django REST Framework para crear APIs
\item Base de datos MySQL para guardar información estructurada
\item Autenticación JWT para seguridad en las conexiones
\item Arquitectura por capas: modelos, serializadores, vistas y servicios
\item La IA fue clave para crear modelos Django con relaciones complejas, validaciones personalizadas y documentación automática de las APIs
\end{itemize}

Parte web (SPA con React/TypeScript):
\begin{itemize}
\item React 18 con TypeScript para desarrollo seguro
\item Vite para compilar y empaquetar rápido
\item Tailwind CSS para estilos que se adaptan a cualquier pantalla
\item peticiones fetch para conectar con APIs y React Router para navegación
\item Patrón de diseño atómico con componentes reutilizables
\item La IA ayudó a crear componentes, formularios con validación, tablas con filtros y optimizaciones de rendimiento
\end{itemize}

Aplicación móvil (con .NET MAUI):
\begin{itemize}
\item .NET MAUI para desarrollo nativo en Android
\item Patrón MVVM con CommunityToolkit.Mvvm
\item Consumo de APIs REST 
\item La IA asistió en la generación de ViewModels observables, implementación de comandos para interacción UI, creación de layouts XAML responsivos y manejo de navegación entre páginas
\end{itemize}

\section{Desarrollo Ágil con Integración de IA}

\begin{figure}[htbp]
\centering
\includegraphics[width=\columnwidth]{graphics/ciclo_desarrollo_ia.pdf}
\caption{Ciclo de desarrollo asistido por IA - Autogestión SENA}
\label{fig:ciclo_ia}
\end{figure}

El proceso de desarrollo se estructuró en sprints de 2 semanas, con reuniones diarias de 30-40 minutos, revisiones semanales y retrospectivas. La integración de IA se realizó de manera estratégica en las siguientes áreas:
\begin{itemize}
\item Generación de código base para modelos Django y serializadores
\item Creación de componentes React reutilizables
\item Implementación de lógica de negocio compleja
\item Generación de pruebas unitarias y de integración
\item Documentación técnica y explicaciones de algoritmos
\item Resolución de bugs y optimización de código
\item Análisis de requisitos y diseño de arquitectura
\item Implementación de patrones de seguridad y autenticación
\item Optimización de consultas de base de datos
\item Configuración de entornos de desarrollo y despliegue
\end{itemize}

\subsection{Proceso de Integración de IA en el Desarrollo}

La integración de IA se realizó de manera sistemática siguiendo un marco de trabajo estructurado que garantizó la calidad y consistencia del código generado. El proceso incluyó:

\textbf{Fase de Análisis y Planificación:}
\begin{itemize}
\item Evaluación de la complejidad de cada componente del sistema
\item Definición de criterios de calidad para el código generado por IA
\item Establecimiento de protocolos de revisión y validación
\item Identificación de áreas críticas que requerían intervención humana
\end{itemize}

\textbf{Fase de Generación Asistida:}
\begin{itemize}
\item Uso de prompts específicos y contextualizados para cada tipo de componente
\item Iteración entre generación automática y refinamiento manual
\item Validación continua de la funcionalidad generada
\item Documentación automática de decisiones de diseño tomadas
\end{itemize}

\textbf{Fase de Integración y Testing:}
\begin{itemize}
\item Integración gradual de componentes generados por IA
\item Ejecución de pruebas unitarias y de integración automatizadas
\item Validación de compatibilidad con el sistema existente
\item Optimización de rendimiento y corrección de bugs identificados
\end{itemize}

Esta metodología estructurada permitió maximizar los beneficios de la IA mientras se mantenía el control de calidad y la capacidad de intervención humana cuando era necesario.

\subsection{Desafíos Técnicos y Soluciones Implementadas}

Durante el proceso de desarrollo asistido por IA, se enfrentaron varios desafíos técnicos que requirieron soluciones innovadoras:

\textbf{Consistencia del Código:}
La IA generaba código con diferentes estilos y convenciones. Se implementó un sistema de linting automático y revisiones de código estandarizadas para asegurar la consistencia.

\textbf{Validación de Lógica de Negocio:}
La IA podía generar código funcional pero no siempre capturaba la lógica específica del dominio SENA. Se establecieron sesiones de revisión detallada con expertos del dominio para validar cada componente crítico.

\textbf{Integración de Tecnologías Heterogéneas:}
El sistema combina Python/Django, React/TypeScript y .NET MAUI. Se desarrolló una estrategia de integración que permitió la comunicación fluida entre estos componentes, utilizando APIs REST estandarizadas y protocolos de comunicación bien definidos.

\textbf{Gestión de Dependencias y Versiones:}
Se implementó un sistema de control de versiones riguroso para todas las dependencias, asegurando la compatibilidad entre los diferentes componentes del sistema.

\begin{figure}[htbp]
\centering
\includegraphics[width=\columnwidth]{graphics/comparacion_metodologias.pdf}
\caption{Comparación de Metodologías Ágiles}
\label{fig:comparacion_metodologias}
\end{figure}

La IA se utilizó en la gran mayoría del proyecto, desde la conceptualización inicial hasta el despliegue final, representando aproximadamente el 70\% de las tareas de desarrollo. Esto incluyó la generación automática de código, refactorización sistemática, debugging asistido y documentación técnica, permitiendo al equipo enfocarse en la lógica de negocio específica del SENA.

\subsection{Medición del Impacto de la IA}

Para cuantificar el impacto de la integración de IA en el proyecto, se implementaron métricas específicas:

\textbf{Métricas de Productividad:}
\begin{itemize}
\item Líneas de código generadas por IA vs. código escrito manualmente
\item Tiempo invertido en debugging y resolución de errores
\item Número de iteraciones necesarias para completar funcionalidades
\item Velocidad de desarrollo medida en story points por sprint
\end{itemize}

\textbf{Métricas de Calidad:}
\begin{itemize}
\item Número de bugs encontrados en revisiones de código
\item Cobertura de pruebas automatizadas
\item Cumplimiento de estándares de codificación
\item Mantenibilidad del código medida por complejidad ciclomática
\end{itemize}

\textbf{Métricas de Eficiencia:}
\begin{itemize}
\item Reducción en tiempo de desarrollo para tareas repetitivas
\item Mejora en la consistencia de la arquitectura del sistema
\item Aceleración en el aprendizaje del equipo
\item Reducción de costos operativos a largo plazo
\end{itemize}

Estos indicadores permitieron demostrar objetivamente los beneficios de la integración de IA, justificando su uso extendido en el proyecto y estableciendo un precedente para futuros desarrollos.