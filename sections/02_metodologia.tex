El proyecto se reestructuró siguiendo una metodología ágil llamada Scrum, durante 4 meses divididos en 7 sprints (ciclos de trabajo). Esta forma de trabajar nos dio flexibilidad para escuchar a los usuarios reales (instructores y coordinadores del SENA) y cambiar el proyecto según sus necesidades.

\subsection{Análisis de Requisitos y Diseño del Sistema}

Empezamos recopilando todo lo que el sistema debía hacer, identificando cinco tipos de usuarios: coordinadores, instructores, Aprendices, administradores y operadores de Sofia plus. Las funciones incluyeron manejar usuarios con seguridad, asignar instructores automática o manualmente, ver horarios, hacer reportes y tener interfaces que funcionen en cualquier dispositivo. La IA ayudó mucho aquí, creando diagramas complejos de relaciones entre datos, flujos de navegación y sugerencias de diseño que aceleraron el trabajo inicial.

El modelo de datos se organizó alrededor de elementos clave como usuarios, roles, Aprendices, instructores, asignaciones y seguimientos, con diferentes tipos de conexiones entre ellos. La IA ayudó a organizar bien la base de datos MySQL, sugiriendo formas de guardar los datos de manera eficiente y segura.

\subsection{Arquitectura Técnica}

El sistema se rediseñó siguiendo reglas de organización clara y capacidad de crecer, usando un patrón de capas basado en módulos que hace fácil mantener y actualizar el software. Tomamos la estructura existente y la mejoramos completamente con tecnologías más modernas y mejores prácticas.

\textbf{Parte back-end (API REST con Django/Python):}
- Django 4.2 con Django REST Framework para crear APIs
- Base de datos MySQL para guardar información estructurada
- Autenticación JWT para seguridad en las conexiones
- Arquitectura por capas: modelos, serializadores, vistas y servicios
- La IA fue clave para crear modelos Django con relaciones complejas, validaciones personalizadas y documentación automática de las APIs

\textbf{Parte web (SPA con React/TypeScript):}
- React 18 con TypeScript para desarrollo seguro
- Vite para compilar y empaquetar rápido
- Tailwind CSS para estilos que se adaptan a cualquier pantalla
- peticiones fetch para conectar con APIs y React Router para navegación
- Patrón de diseño atómico con componentes reutilizables
- La IA ayudó a crear componentes, formularios con validación, tablas con filtros y optimizaciones de rendimiento

\textbf{Aplicación móvil (con .NET MAUI):}
- .NET MAUI para desarrollo nativo en Android
- Patrón MVVM con CommunityToolkit.Mvvm
- Consumo de APIs REST 
- La IA asistió en la generación de ViewModels observables, implementación de comandos para interacción UI, creación de layouts XAML responsivos y manejo de navegación entre páginas

\subsection{Desarrollo Ágil con Integración de IA}

El proceso de desarrollo se estructuró en sprints de 2 semanas, con reuniones diarias de 30-40 minutos, revisiones semanales y retrospectivas. La integración de IA se realizó de manera estratégica en las siguientes áreas:
- Generación de código base para modelos Django y serializadores
- Creación de componentes React reutilizables
- Implementación de lógica de negocio compleja
- Generación de pruebas unitarias y de integración
- Documentación técnica y explicaciones de algoritmos
- Resolución de bugs y optimización de código

La IA se utilizó en la gran mayoría del proyecto, desde la conceptualización inicial hasta el despliegue final, representando aproximadamente el 70\% de las tareas de desarrollo. Esto incluyó la generación automática de código, refactorización sistemática, debugging asistido y documentación técnica, permitiendo al equipo enfocarse en la lógica de negocio específica del SENA.
\end{itemize}

\subsection{Diseño Técnico}

\subsubsection{Arquitectura del Sistema}
Se adoptó una arquitectura de tres capas con separación clara de responsabilidades:

\begin{enumerate}
\item \textbf{Capa de Presentación}: Aplicaciones web y móvil como interfaces de usuario.
\item \textbf{Capa de Aplicación}: API RESTful que implementa lógica de negocio.
\item \textbf{Capa de Datos}: Base de datos relacional con diseño optimizado.
\end{enumerate}

\subsubsection{Selección Tecnológica}
La selección de tecnologías se basó en criterios objetivos:

\begin{itemize}
\item \textbf{Madurez y Comunidad}: Tecnologías con amplio soporte y documentación.
\item \textbf{Alineación Institucional}: Compatibilidad con infraestructura existente del SENA.
\item \textbf{Curva de Aprendizaje}: Tecnologías accesibles para estudiantes en formación.
\item \textbf{Escalabilidad}: Capacidad para crecer con las necesidades institucionales.
\end{itemize}

\subsection{Validación y Evaluación}

\subsubsection{Estrategia de Testing}
Se implementó una pirámide de testing completa:

\begin{itemize}
\item \textbf{Pruebas Unitarias}: Cobertura >80\% en código backend.
\item \textbf{Pruebas de Integración}: Validación de flujos completos.
\item \textbf{Pruebas de Aceptación}: Validación con usuarios finales.
\item \textbf{Pruebas de Rendimiento}: Simulación de carga institucional.
\end{itemize}

\subsubsection{Métricas de Evaluación}
Se definieron métricas cuantitativas y cualitativas:

\begin{enumerate}
\item \textbf{Eficiencia}: Reducción en tiempo de asignación (meta: 60\%).
\item \textbf{Satisfacción}: Encuestas de usabilidad (escala Likert 1-5).
\item \textbf{Calidad}: Tasa de errores en asignaciones (meta: <5\%).
\item \textbf{Adopción}: Porcentaje de usuarios activos (meta: 80\%).
\end{enumerate}

\subsection{Consideraciones Éticas}

El estudio siguió principios éticos de investigación:
\begin{itemize}
\item Consentimiento informado de todos los participantes.
\item Anonimización de datos personales en reportes.
\item Transparencia en el uso de datos para desarrollo.
\item Confidencialidad en información institucional sensible.
\end{itemize}

\subsection{Limitaciones Metodológicas}

Se reconocen las siguientes limitaciones:
\begin{itemize}
\item Alcance geográfico limitado a una regional del SENA.
\item Tiempo de estudio restringido a 4 meses.
\item Dependencia de disponibilidad de participantes.
\item Posible sesgo de selección en participantes voluntarios.
\end{itemize}

Estas limitaciones se mitigaron mediante triangulación de métodos y validación cruzada de hallazgos.
