\section{Metodología}
\label{sec:metodologia}

Este estudio empleó un enfoque mixto que combina métodos cualitativos y cuantitativos, integrando investigación aplicada con desarrollo de software. La metodología se fundamenta en el paradigma de investigación-acción, donde el proceso de desarrollo del software se convierte en el medio para investigar y mejorar prácticas institucionales.

\subsection{Diseño de Investigación}

Se adoptó un diseño de estudio de caso instrumental \cite{stake1995case}, donde el desarrollo de "Autogestión SENA" sirve como instrumento para explorar la colaboración humano-IA en contextos de formación técnica colombiana. El estudio combina elementos de investigación aplicada con desarrollo de producto software.

\subsubsection{Participantes}
El estudio involucró múltiples grupos de interés del SENA:

\begin{itemize}
\item \textbf{Coordinadores Regionales} (n=12): Responsables de asignación de instructores en diferentes regionales.
\item \textbf{Instructores} (n=25): Profesionales técnicos con experiencia docente entre 2-15 años.
\item \textbf{Equipo de Desarrollo} (n=4): Estudiantes de ingeniería de sistemas en formación práctica.
\item \textbf{Directivos Institucionales} (n=3): Representantes de la dirección nacional del SENA.
\end{itemize}

\subsubsection{Instrumentos de Recolección de Datos}
Se emplearon múltiples instrumentos para triangulación de datos:

\begin{enumerate}
\item \textbf{Entrevistas Semiestructuradas}: 15 entrevistas individuales con coordinadores e instructores, con duración promedio de 45 minutos. El protocolo incluyó preguntas sobre procesos actuales, dolor points y expectativas del sistema.
\item \textbf{Encuestas Cualitativas}: Aplicadas a 40 potenciales usuarios, enfocadas en requerimientos funcionales y no funcionales.
\item \textbf{Observación Participante}: El equipo de desarrollo participó en reuniones de coordinación reales durante 8 semanas.
\item \textbf{Análisis Documental}: Revisión de procedimientos institucionales, reportes de asignación y estadísticas históricas.
\end{enumerate}

\subsection{Metodología de Desarrollo}

\subsubsection{Enfoque Ágil Adaptado}
Se implementó Scrum adaptado al contexto educativo, con las siguientes modificaciones:

\begin{itemize}
\item \textbf{Sprints de 3 semanas}: Tiempo extendido para permitir aprendizaje profundo de tecnologías.
\item \textbf{Daily Standups Educativos}: Incluían componentes de mentoría técnica.
\item \textbf{Revisiones con Stakeholders}: Sesiones semanales con usuarios finales para validación.
\item \textbf{Retrospectivas de Aprendizaje}: Enfoque en lecciones técnicas y pedagógicas.
\end{itemize}

\subsubsection{Integración de IA en el Proceso}
La IA se incorporó como herramienta colaborativa en todas las fases:

\begin{itemize}
\item \textbf{Planificación}: Generación de historias de usuario y criterios de aceptación.
\item \textbf{Diseño}: Creación de diagramas UML y mockups de interfaz.
\item \textbf{Desarrollo}: Generación de código boilerplate y sugerencias de implementación.
\item \textbf{Pruebas}: Creación de casos de prueba y análisis de cobertura.
\item \textbf{Documentación}: Generación automática de documentación técnica.
\end{itemize}

\subsection{Diseño Técnico}

\subsubsection{Arquitectura del Sistema}
Se adoptó una arquitectura de tres capas con separación clara de responsabilidades:

\begin{enumerate}
\item \textbf{Capa de Presentación}: Aplicaciones web y móvil como interfaces de usuario.
\item \textbf{Capa de Aplicación}: API RESTful que implementa lógica de negocio.
\item \textbf{Capa de Datos}: Base de datos relacional con diseño optimizado.
\end{enumerate}

\subsubsection{Selección Tecnológica}
La selección de tecnologías se basó en criterios objetivos:

\begin{itemize}
\item \textbf{Madurez y Comunidad}: Tecnologías con amplio soporte y documentación.
\item \textbf{Alineación Institucional}: Compatibilidad con infraestructura existente del SENA.
\item \textbf{Curva de Aprendizaje}: Tecnologías accesibles para estudiantes en formación.
\item \textbf{Escalabilidad}: Capacidad para crecer con las necesidades institucionales.
\end{itemize}

\subsection{Validación y Evaluación}

\subsubsection{Estrategia de Testing}
Se implementó una pirámide de testing completa:

\begin{itemize}
\item \textbf{Pruebas Unitarias}: Cobertura >80\% en código backend.
\item \textbf{Pruebas de Integración}: Validación de flujos completos.
\item \textbf{Pruebas de Aceptación}: Validación con usuarios finales.
\item \textbf{Pruebas de Rendimiento}: Simulación de carga institucional.
\end{itemize}

\subsubsection{Métricas de Evaluación}
Se definieron métricas cuantitativas y cualitativas:

\begin{enumerate}
\item \textbf{Eficiencia}: Reducción en tiempo de asignación (meta: 60\%).
\item \textbf{Satisfacción}: Encuestas de usabilidad (escala Likert 1-5).
\item \textbf{Calidad}: Tasa de errores en asignaciones (meta: <5\%).
\item \textbf{Adopción}: Porcentaje de usuarios activos (meta: 80\%).
\end{enumerate}

\subsection{Consideraciones Éticas}

El estudio siguió principios éticos de investigación:
\begin{itemize}
\item Consentimiento informado de todos los participantes.
\item Anonimización de datos personales en reportes.
\item Transparencia en el uso de datos para desarrollo.
\item Confidencialidad en información institucional sensible.
\end{itemize}

\subsection{Limitaciones Metodológicas}

Se reconocen las siguientes limitaciones:
\begin{itemize}
\item Alcance geográfico limitado a una regional del SENA.
\item Tiempo de estudio restringido a 4 meses.
\item Dependencia de disponibilidad de participantes.
\item Posible sesgo de selección en participantes voluntarios.
\end{itemize}

Estas limitaciones se mitigaron mediante triangulación de métodos y validación cruzada de hallazgos.
