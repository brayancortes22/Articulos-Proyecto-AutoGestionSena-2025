El proyecto se reestructuró siguiendo una metodología ágil llamada Scrum, durante 4 meses divididos en 7 sprints (ciclos de trabajo). Esta forma de trabajar nos dio flexibilidad para escuchar a los usuarios reales (instructores y coordinadores del SENA) y cambiar el proyecto según sus necesidades.

\section{Análisis de Requisitos y Diseño del Sistema}

Empezamos recopilando todo lo que el sistema debía hacer, identificando cinco tipos de usuarios: coordinadores, instructores, Aprendices, administradores y operadores de Sofia plus. Las funciones incluyeron manejar usuarios con seguridad, asignar instructores automática o manualmente, ver horarios, hacer reportes y tener interfaces que funcionen en cualquier dispositivo. La IA ayudó mucho aquí, creando diagramas complejos de relaciones entre datos, flujos de navegación y sugerencias de diseño que aceleraron el trabajo inicial.

El modelo de datos se organizó alrededor de elementos clave como usuarios, roles, Aprendices, instructores, asignaciones y seguimientos, con diferentes tipos de conexiones entre ellos. La IA ayudó a organizar bien la base de datos MySQL, sugiriendo formas de guardar los datos de manera eficiente y segura.

\section{Arquitectura Técnica}

El sistema se rediseñó siguiendo reglas de organización clara y capacidad de crecer, usando un patrón de capas basado en módulos que hace fácil mantener y actualizar el software. Tomamos la estructura existente y la mejoramos completamente con tecnologías más modernas y mejores prácticas.

Parte back-end (API REST con Django/Python):
\begin{itemize}
\item Django 4.2 con Django REST Framework para crear APIs
\item Base de datos MySQL para guardar información estructurada
\item Autenticación JWT para seguridad en las conexiones
\item Arquitectura por capas: modelos, serializadores, vistas y servicios
\item La IA fue clave para crear modelos Django con relaciones complejas, validaciones personalizadas y documentación automática de las APIs
\end{itemize}

Parte web (SPA con React/TypeScript):
\begin{itemize}
\item React 18 con TypeScript para desarrollo seguro
\item Vite para compilar y empaquetar rápido
\item Tailwind CSS para estilos que se adaptan a cualquier pantalla
\item peticiones fetch para conectar con APIs y React Router para navegación
\item Patrón de diseño atómico con componentes reutilizables
\item La IA ayudó a crear componentes, formularios con validación, tablas con filtros y optimizaciones de rendimiento
\end{itemize}

Aplicación móvil (con .NET MAUI):
\begin{itemize}
\item .NET MAUI para desarrollo nativo en Android
\item Patrón MVVM con CommunityToolkit.Mvvm
\item Consumo de APIs REST 
\item La IA asistió en la generación de ViewModels observables, implementación de comandos para interacción UI, creación de layouts XAML responsivos y manejo de navegación entre páginas
\end{itemize}

\section{Desarrollo Ágil con Integración de IA}

El proceso de desarrollo se estructuró en sprints de 2 semanas, con reuniones diarias de 30-40 minutos, revisiones semanales y retrospectivas. La integración de IA se realizó de manera estratégica en las siguientes áreas:
\begin{itemize}
\item Generación de código base para modelos Django y serializadores
\item Creación de componentes React reutilizables
\item Implementación de lógica de negocio compleja
\item Generación de pruebas unitarias y de integración
\item Documentación técnica y explicaciones de algoritmos
\item Resolución de bugs y optimización de código
\end{itemize}

La IA se utilizó en la gran mayoría del proyecto, desde la conceptualización inicial hasta el despliegue final, representando aproximadamente el 70\% de las tareas de desarrollo. Esto incluyó la generación automática de código, refactorización sistemática, debugging asistido y documentación técnica, permitiendo al equipo enfocarse en la lógica de negocio específica del SENA.