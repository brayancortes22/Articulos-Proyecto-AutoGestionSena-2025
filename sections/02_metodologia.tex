\section{Metodología: Diseño y Planificación}

El proyecto se ejecutó bajo una metodología ágil (Scrum) durante un periodo total de cuatro meses. A continuación se detallan las actividades principales realizadas durante la planificación y diseño:

\subsection{Diseño de base de datos y modelo del dominio}
Se diseñaron las entidades principales (Usuario, Persona, Rol, Aprendiz, Instructor, etc.) y las relaciones entre ellas utilizando diagramas en Draw.io. Además, se consultó IA para identificar redundancias, sugerir índices y mejorar la normalización.

\subsection{Prototipado UX/UI}
Los mockups y flujos de navegación se elaboraron en Figma. La IA se empleó para generar variaciones visuales, recomendaciones de accesibilidad y propuestas de interacción, lo cual ayudó a generar pruebas de usabilidad tempranas.

\subsection{Planificación y decisiones arquitectónicas}
La IA contribuyó a comparar alternativas de arquitectura (monolito modular vs microservicios) y a documentar decisiones de diseño. Se contrastaron estas recomendaciones con los requisitos del SENA y la disponibilidad del equipo.

\subsection{Marco de trabajo y cadence}
El equipo trabajó con sprints y reuniones diarias (daily standup), revisiones y retrospectivas. En total se ejecutaron 7 sprints durante el periodo del proyecto, con una cadencia aproximada de 2 sprints por mes.

Se priorizaron funcionalidades críticas, y la planificación se ajustó iterativamente para integrar retroalimentación de los instructores y usuarios finales.
