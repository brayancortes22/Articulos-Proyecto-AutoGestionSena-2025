\appendix
Herramientas y repositorio

Durante el desarrollo del proyecto se emplearon asistentes de código y modelos conversacionales para apoyar tareas de generación de código, debugging y consultas técnicas, tal como se indica en el texto del artículo.

Repositorio del proyecto: el código fuente y la documentación están disponibles en el repositorio del trabajo, que incluye el modelo entidad-relación, mockups en Figma, el back-end en Django/Python y el front-end web.

\subsection{Ejemplo de endpoint (Django Rest Framework)}
\begin{lstlisting}[language=python, caption={Ejemplo de endpoint para crear sesiones con validación básica}]
from rest_framework import viewsets
from rest_framework.response import Response
from rest_framework import status
from .serializers import SesionSerializer

class SesionViewSet(viewsets.ViewSet):
	def create(self, request):
		serializer = SesionSerializer(data=request.data)
		if serializer.is_valid():
			sesion = serializer.save()
			return Response(SesionSerializer(sesion).data, status=status.HTTP_201_CREATED)
		return Response(serializer.errors, status=status.HTTP_400_BAD_REQUEST)
\end{lstlisting}

\subsection{Comando de ejemplo para despliegue con Docker Compose}
\begin{lstlisting}
docker-compose pull
docker-compose up --build -d
docker-compose exec web python manage.py migrate
docker-compose exec web python manage.py collectstatic --no-input
\end{lstlisting}

