En este artículo contamos la experiencia de reestructurar el proyecto "Autogestión SENA", una aplicación completa (web y móvil) que ayuda a asignar instructores en los centros de formación del SENA. El proceso incluyó tomar un proyecto existente, mejorarlo completamente, rediseñar, reprogramar y poner en marcha el sistema actualizado, usando mucho la inteligencia artificial para acelerar el trabajo. La solución usa tecnologías modernas como Python/Django para la parte back-end, React/TypeScript para la web, .NET MAUI para el móvil Android, MySQL para guardar datos, y Docker para el despliegue. También analizamos los problemas que encontramos, las soluciones que usamos y lo que aprendimos trabajando juntos humanos y máquinas.