\begin{abstract}
“Autogestión SENA”: Desarrollo Full-Stack Acelerado mediante Colaboración Humano-IA

Nuestro equipo desarrolló el proyecto “Autogestión SENA”, una aplicación completa que incluye versión web y móvil, con el objetivo principal de simplificar y agilizar el proceso de asignación de instructores en los centros de formación del SENA. Antes, este trámite se hacía con correos electrónicos, hojas de cálculo y muchas llamadas, lo que generaba errores, retrasos y bastante estrés tanto para coordinadores como para los mismos instructores.
\end{abstract}