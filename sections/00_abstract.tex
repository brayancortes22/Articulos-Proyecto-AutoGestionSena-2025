\begin{abstract}
En este documento presentamos el desarrollo y la experiencia de construcción del proyecto “Autogestión SENA”, una aplicación full-stack (web y móvil) creada para optimizar la asignación de instructores en centros de formación. El proceso incluyó la recolección de requisitos, diseño, implementación y despliegue, y se apoyó de manera significativa en herramientas de Inteligencia Artificial (IA) para acelerar fases de diseño, generación de código, refactorización, pruebas y despliegue. Además, se documentan las dificultades encontradas, las estrategias aplicadas y las lecciones aprendidas.
\end{abstract}