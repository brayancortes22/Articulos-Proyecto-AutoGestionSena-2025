Este proyecto abordó el desarrollo de ``Autogestión SENA'', una aplicación web/mobile vital para optimizar la asignación de instructores de seguimiento dentro de la institución. El proceso de desarrollo fue exhaustivo, abarcando la recolección de artefactos (SRS, historias de usuario, entrevistas), el diseño de arquitectura (web/mobile, base de datos) y la implementación full-stack (Front-end y Back-end con Python/Django). Lo distintivo de esta experiencia fue la integración estratégica de herramientas de Inteligencia Artificial (IA). La IA no solo actuó como un asistente de código (generando endpoints, debuggeando), sino que también se consolidó como un agente de aprendizaje colaborativo en tiempo real, permitiendo al equipo sintetizar información de investigación y validar diseños complejos con una eficiencia sin precedentes. Este enfoque demostró cómo la IA puede potenciar la productividad y elevar la calidad del aprendizaje técnico colaborativo en proyectos de desarrollo de software.