Nuestro proyecto no es algo aislado. Se conecta con investigaciones que ya muestran cómo la IA está ayudando en dos áreas:
1.	En la educación: Como herramienta para aprender a programar y desarrollar proyectos.
2.	En el desarrollo de software: Para escribir código más rápido y con menos errores.
Con "Autogestión SENA" demostramos en la práctica lo que otros estudios ya decían: la IA sirve como un gran compañero de aprendizaje y desarrollo, especialmente en entornos formativos como el SENA.

\subsection{IA en la Enseñanza de Programación}

Varios estudios confirman que la IA puede transformar significativamente la enseñanza de la programación. Hidalgo et al. [1] destacan cómo las herramientas de IA facilitan el aprendizaje activo y colaborativo, validando nuestra experiencia donde la IA aceleró el desarrollo de componentes complejos en múltiples lenguajes (Python, JavaScript, C\#). De manera similar, Terán [2] demuestra que el uso ético de ChatGPT mejora el rendimiento estudiantil, lo cual se refleja en nuestro proyecto donde la IA contribuyó al 70\% del código generado mientras mantenía estándares de calidad.

\subsection{Colaboración Humano-IA en Desarrollo de Software}

La literatura reciente enfatiza la importancia de la colaboración humano-IA. Pincay [9] describe cómo la IA automatiza tareas repetitivas y genera código eficiente, hallazgo que se confirma en nuestro proyecto donde la IA manejó código repetitivo, refactorización y optimización. Baltazar [16] subraya que las herramientas de IA deben complementarse con supervisión humana, principio que aplicamos rigurosamente en todas las fases del desarrollo.

\subsection{Implicaciones Éticas y Pedagógicas}

Los estudios consultados coinciden en la necesidad de un uso ético de la IA. Sánchez-Vera [5] argumenta que la IA debe replantear las metodologías educativas, perspectiva que se materializa en nuestro proyecto donde la IA no solo aceleró el desarrollo técnico sino que también facilitó el aprendizaje de nuevas tecnologías por parte del equipo. Mejías et al. [3] destacan que la IA en educación requiere involucramiento activo de los profesionales, enfoque que adoptamos al integrar la IA desde las fases iniciales del proyecto.

\subsection{IA en Contextos Educativos Institucionales}

Particularmente relevante para instituciones como el SENA, los estudios de Cadena y Juárez [8] muestran cómo los agentes conversacionales mejoran el aprendizaje de programación, validando nuestro uso de IA para resolver problemas técnicos complejos. De Sande y Ramos [12] demuestran el impacto positivo de asistentes basados en IA en la enseñanza universitaria, hallazgo que se refleja en la reducción del tiempo de desarrollo y mejora de la calidad del código en nuestro proyecto.

\subsection{Contribución de Esta Investigación}

Esta experiencia contribuye a la literatura existente al proporcionar evidencia empírica de la efectividad de la colaboración humano-IA en un proyecto real de desarrollo de software educativo. Mientras que muchos estudios se centran en aspectos teóricos o experimentos controlados, nuestro trabajo documenta la aplicación práctica en un contexto institucional colombiano, validando los beneficios identificados en la literatura mientras destaca desafíos específicos del desarrollo en entornos educativos.