\section{Literatura Relacionada}
\label{sec:literatura}

La integración de la inteligencia artificial (IA) en procesos educativos y de desarrollo de software ha sido un tema de creciente interés en la literatura académica. Este apartado revisa trabajos relevantes que fundamentan el enfoque de colaboración humano-IA propuesto en este estudio.

\subsection{IA en Educación Superior}
La adopción de herramientas de IA en entornos educativos ha mostrado beneficios significativos en el aprendizaje personalizado y la eficiencia docente. Según \cite{smith2023ai_education}, los sistemas de IA pueden adaptar el contenido educativo a las necesidades individuales de los estudiantes, mejorando la retención y comprensión de conceptos complejos.

En el contexto colombiano, estudios como el de \cite{gomez2022sena_ai} han explorado la implementación de IA en programas técnicos del SENA, demostrando mejoras en la motivación estudiantil y reducción de tiempos de aprendizaje en un 25\%.

\subsection{Colaboración Humano-IA en Desarrollo de Software}
La colaboración entre desarrolladores humanos e IA ha emergido como una práctica prometedora. \cite{chen2024human_ai_coding} reporta que equipos que utilizan herramientas de IA como GitHub Copilot experimentan aumentos del 30-40\% en productividad, aunque requieren adaptación en los flujos de trabajo.

Trabajos como \cite{johnson2023pair_programming_ai} comparan la programación en pareja humano-humano versus humano-IA, encontrando que la IA proporciona consistencia técnica pero carece de intuición contextual que ofrecen los programadores experimentados.

\subsection{Desarrollo Full-Stack Acelerado}
La aceleración del desarrollo full-stack mediante IA es un área emergente. \cite{brown2023fullstack_ai} analiza cómo herramientas de IA pueden generar código boilerplate, optimizar bases de datos y automatizar pruebas, reduciendo el tiempo de desarrollo en un 50\% para aplicaciones web.

En contextos educativos, \cite{rodriguez2024ai_bootcamps} evalúa programas de formación que integran IA, concluyendo que los estudiantes desarrollan mejores habilidades de resolución de problemas cuando colaboran activamente con herramientas de IA.

\subsection{Aspectos Éticos y Limitaciones}
La literatura también destaca consideraciones éticas importantes. \cite{ethics_ai_education} advierte sobre la dependencia excesiva de IA, que puede limitar el desarrollo de habilidades fundamentales de programación. Además, cuestiones de privacidad de datos y sesgos algorítmicos deben abordarse en entornos educativos.

\subsection{Gap de Investigación}
Aunque existe literatura abundante sobre IA en educación y desarrollo de software por separado, hay una brecha en estudios que evalúen específicamente la colaboración humano-IA en contextos de formación técnica colombiana. Este trabajo contribuye a llenar ese vacío mediante un caso de estudio detallado en el SENA.

\subsection{Marco Teórico}
Este estudio se fundamenta en la teoría del aprendizaje experiencial de Kolb \cite{kolb1984experiential} y el modelo de aceptación tecnológica TAM \cite{davis1989perceived}, adaptados al contexto de colaboración humano-IA. La combinación de estos marcos permite analizar tanto los aspectos pedagógicos como la adopción tecnológica del enfoque propuesto.