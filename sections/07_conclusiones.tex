\section{Conclusiones y trabajo futuro}
Este proyecto demostró que la integración estratégica de herramientas de Inteligencia Artificial en el desarrollo de software puede transformar significativamente la productividad y el aprendizaje colaborativo en equipos de desarrollo. La experiencia con ``Autogestión SENA'' reveló dos contribuciones principales:

\subsection{Contribuciones principales}
\begin{enumerate}
    \item \textbf{Resolución inmediata de bloqueos técnicos:} La disponibilidad 24/7 de asistencia mediante IA eliminó los tiempos de espera tradicionalmente asociados con dudas técnicas, permitiendo un flujo de trabajo continuo.
    
    \item \textbf{Optimización del tiempo de desarrollo:} La automatización de tareas mecánicas y repetitivas liberó tiempo valioso para que el equipo se enfocara en aspectos creativos y de diseño que requieren pensamiento crítico humano.
\end{enumerate}

\subsection{Lecciones aprendidas}
La IA funcionó efectivamente como complemento del talento humano, no como reemplazo. La arquitectura, la lógica de negocio y las decisiones de diseño permanecieron bajo control humano, mientras que la IA aceleró la implementación de componentes técnicos repetitivos \cite{cadena2023ensenanza}.

\subsection{Trabajo futuro}
El trabajo futuro explorará:
\begin{itemize}
    \item La medición cuantitativa del impacto de IA en métricas de productividad específicas
    \item La adaptación de este enfoque a otros contextos educativos dentro del SENA
    \item El desarrollo de guías de mejores prácticas para la integración ética de IA en entornos de aprendizaje de programación \cite{teran2023implementacion}
    \item La evaluación del impacto a largo plazo en las competencias técnicas adquiridas por los estudiantes \cite{pincay2025aplicaciones}
\end{itemize}

Este enfoque puede servir como modelo para otras instituciones educativas que buscan acelerar el aprendizaje técnico mediante la colaboración estratégica entre humanos e IA.