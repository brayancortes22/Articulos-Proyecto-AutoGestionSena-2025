\section{Conclusiones}
La colaboración humano-IA aplicada en el proyecto "Autogestión SENA" aceleró tareas técnicas y mejoró el proceso de aprendizaje del equipo. La IA aportó principalmente en:
\begin{itemize}
	\item Reducción de tareas repetitivas y generación de esqueletos de código.
	\item Soporte para depuración rápida y explicaciones de funciones complejas.
	\item Generación de pruebas y scripts de despliegue que facilitaron la entrega continua.
\end{itemize}

Sin embargo, la IA puede introducir suposiciones incorrectas, por lo que las sugerencias deben ser vigiladas y probadas por el equipo humano. En consecuencia, la integración de IA resultó ser complementaria y no sustitutiva.

\subsection{Trabajo futuro}
Se recomiendan las siguientes líneas de trabajo para consolidar y mejorar el enfoque:
\begin{itemize}
	\item Automatizar CI/CD con pruebas E2E y análisis de calidad continua.
	\item Medir cuantitativamente el ahorro de horas y la mejora en la calidad del código.
	\item Investigar modelos de IA personalizados para el dominio SENA que reduzcan la necesidad de ajustes manuales.
\end{itemize}

\subsection{Cierre}
La experiencia del proyecto muestra que el uso responsable de IA puede potenciar la productividad y la calidad educativa en proyectos de formación técnica como el SENA. Implementar controles de seguridad, revisión humana y pruebas automatizadas son pasos esenciales para escalar este enfoque.