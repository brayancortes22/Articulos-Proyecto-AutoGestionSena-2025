\section{Conclusiones}
\label{sec:conclusiones}

El uso de herramientas de IA en el desarrollo del proyecto "Autogestión SENA" mostró un aporte significativo en la aceleración de tareas repetitivas, la generación de esqueletos, la resolución de dudas de programación y la documentación inicial de la solución. Sin embargo, las recomendaciones y el código generado por IA requieren revisión humana para adaptarlos al contexto y las políticas del equipo.

Las principales lecciones aprendidas incluyen:
\begin{itemize}
	\item Integrar IA como herramienta de apoyo (no sustituto) para acelerar tareas repetitivas y mejorar la productividad.
	\item Mantener un proceso de revisión humana y tests automatizados para validar propuestas generadas por IA.
	\item Aplicar controles para evitar la exposición de secretos o malas prácticas generadas automáticamente.
\end{itemize}

\section{Bibliografía}
\begin{itemize}
	\item Cadena, O. A., \& Juárez, I. A. (2023). La enseñanza de la programación mediante software educativo especializado y los agentes conversacionales. Interfases, (017), 170-186.
	\item Terán, H. (2023). La implementación de la Inteligencia Artificial en la enseñanza de la programación. Un estudio sobre el uso ético de ChatGPT en el aula. Encuentro Internacional de Educación en Ingeniería.
	\item Pincay, W. D. C. (2025). Aplicaciones de la Inteligencia Artificial en el campo de la programación. Journal TechInnovation, 4(1), 4-12.
\end{itemize}

